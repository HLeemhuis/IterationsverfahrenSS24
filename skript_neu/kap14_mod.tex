\section{Unvollst"andige LU-Zerlegungen}
Eine unvollst"andige LU-Zerlegung ( ILU ) wird ben"otig als Pr"akonditionierer.
Starten wollen wir mit einer zeilenorientierten Variante der Gau"s-Elimination.
Dazu sei $A\in \cnn$ regul"ar.

\begin{alg}[$ikj$-Form der Gau"s Elimination]
~               % um "Algorithmus" aus dem Kasten rauszubekommen
\vspace*{-2\baselineskip}       % um den Leeraum zu entfernen
\begin{algorithm}
  \begin{algorithmic}
    \FOR{$i = 2, \dots ,n$ }
      \FOR{$ k = 1, \dots ,i-1$}
        \STATE $a_{i,k} = a_{i,k}/a_{k,k}$
        \FOR{$ j = k+1, \dots , n$}
          \STATE $a_{i,j} = a_{i,j} - a_{i,k} a_{k,j}$
        \ENDFOR
      \ENDFOR
    \ENDFOR
  \end{algorithmic}
\end{algorithm}
\end{alg}

Nach Terminierung enth"alt das strikte untere Dreieck von $A$ den Faktor $L$;
der Rest ist $U$.
\medskip

{\bf Vorteil} der $ikj$-Form: $L$ und $U$ werden zeilenweise berechnet. Dies ist geeignet f"ur die Datenstruktur bei d"unnbesetzten Matrizen.

\medskip

{\bf Idee bei ILU:} Lasse Operationen in der Gau"s-Elimination weg, so dass die Matrizen weniger gef"ullt sind.

\medskip

\subsection{ILU zu einem vorgegebenen Muster}

\begin{defn}
$E \subseteq \{1,\dots,n \} \times \{1,\dots,n\}$ hei"st {\em Muster}, wenn $\{(i,i) , i=1, \dots ,n\} \subseteq E.$
\end{defn}

\begin{alg}[$ikj$-Form der ILU zum Muster E]
~               % um "Algorithmus" aus dem Kasten rauszubekommen
\vspace*{-2\baselineskip}       % um den Leeraum zu entfernen
\begin{algorithm}
  \begin{algorithmic}
    \FOR{$i = 2, \dots ,n$ }
      \FOR{$ k = 1, \dots ,i-1$}
        \IF{$(i,k) \in E$}
          \STATE $a_{i,k} = a_{i,k}/a_{k,k}$
          \FOR{$ j = k+1, \dots , n$}
            \IF{$(i,j) \in E \wedge (k,j) \in E$}
              \STATE $a_{i,j} = a_{i,j} - a_{i,k} a_{k,j}$
            \ENDIF
          \ENDFOR
        \ENDIF
      \ENDFOR
    \ENDFOR
  \end{algorithmic}
\end{algorithm}
\end{alg}

F"ur $U$ und $L$ in der $LU$-Faktorisierung gilt dann
$$ u_{ij} =
\begin{cases}
  a_{i,j}  & \text{falls } (i,j) \in E \text{ und } j \geq i \\
  0 & \text{sonst}
\end{cases}
$$
und 
$$ l_{ij} =
\begin{cases}
  a_{i,j}  & \text{falls } (i,j) \in E \text{ und } j < i \\
  0 & \text{sonst}.
\end{cases}
$$

\begin{sa}
Sei $A$ eine H-Matrix, dann existiert die ILU f"ur jedes Muster $E$.
\end{sa}

{\bf Beweis:} s.\ Vorlesung "`Parallele Algorithmen WS 2005/06"'.

Nun stellt sich nat"urlich die Frage, was geeignete Muster $E$ sind.

\begin{defn}
Man spricht von ILU(0), falls $ E = \{ (i,j) : a_{ij} \neq 0 \} $.
\end{defn}

{\bf Frage:} Wie findet man weitere, gr"o"sere geeignete Muster?

\medskip

{\bf Idee:} Wir gehen davon aus, dass $ \diag(A) = I $ und $|a_{i,j}| < \varepsilon$
f"ur  $i \neq j$ (z.B. $\varepsilon = \frac 14$ bei Modellproblem I). Dann kann man einen "`F"ull-Level-Index"' auf
folgende Weise vergeben:

\medskip

Initialisierung:
$$
\widehat{lev}(i,j) =
\begin{cases}
   1 & \text{falls } a_{i,j} \neq 0 \\
   \infty & \text{falls } a_{i,j} = 0.
\end{cases}
$$
Bei jedem Schritt der (unvollst"andigen) Gau"s-Elimination wird $\widehat{lev}(i,j)$ 
aufdatiert via
\begin{equation}\label{aufdatierunglevel}
a_{i,j} = a_{i,j} - a_{i,k} a_{k,j} \Rightarrow \widehat{lev}(i,j) = \min \{ \widehat{lev}(i,j),\widehat{lev}(i,k) + \widehat{lev}(k,j) \}
\end{equation}
Geht man davon aus, dass alle $a_{i,i}$ die Gr"o"senordnung 1 haben, so hat
$a_{i,j}$ die Gr"o"senordnung ${\cal O}(\varepsilon^{\widehat{lev}(i,j)})$.
Je gr"o"ser $\widehat{lev}(i,j)$, desto kleiner $a_{i,j}$. In einer ILU
kann man also das Muster so festlegen, dass die Eintr"age mit gro"sem F"ull-Level
auf Null gesetzt werden.

In der Standardterminologie wird $lev(i,j) = \widehat{lev}(i,j)-1$ verwendet, d.h. (\ref{aufdatierunglevel}) wird zu
\begin{equation}\label{aufdatierungstandard}
lev(i,j) = \min \{lev(i,j) , lev(i,k) + lev(k,j) +1 \}.
\end{equation}

In der ILU($p$) werden nun die Positionen in das Muster $E$ "ubernommen, f"ur
die $lev(i,j) \leq p$. Dies ist konsistent mit der Bezeichnung
ILU(0) von vorher.

\begin{alg}[ikj-Form der ILU(p)]
~               % um "Algorithmus" aus dem Kasten rauszubekommen
\vspace*{-2\baselineskip}       % um den Leeraum zu entfernen
\begin{algorithm}
  \begin{algorithmic}
    \FOR{$i = 2, \dots ,n$ }
      \FOR{$ k = 1, \dots ,i-1$}
        \IF{$lev(i,k) \leq p$}
          \STATE $a_{i,k} = a_{i,k}/a_{k,k}$
          \FOR{$ j = k+1, \dots , n$}
            \STATE $lev(i,j) = \min \{lev(i,j),lev(i,k) + lev(k,j) + 1\} $
            \IF{$lev(i,k) \leq p$}
              \STATE $a_{i,j} = a_{i,j} - a_{i,k} a_{k,j}$
            \ENDIF
          \ENDFOR
        \ENDIF
      \ENDFOR
    \ENDFOR
  \end{algorithmic}
\end{algorithm}
\end{alg}

\begin{enumerate}
\item Eine gute Implementierung sollte so sein, dass der Aufwand zur Berechnung der
Level-Indizes nicht gro"s  wird.
\item Das entstehende Muster $E$ ist unabh"angig von den numerischen Werten. Es kann also
eine Datenstruktur von $L$ und $U$ vorweg berechnet werden.
\end{enumerate}

\begin{aufg}
Sei $A$ symmetrisch und $G(A) = (V,E)$ der Graph von $A$.
Zeige:
\begin{itemize}
\item[(i)] Bei der normalen Gau"s-Elimination hat die Matrix $(L\setminus U)$ den
Graphen $\tilde{G} = (V,\tilde{E})$ mit
\begin{multline*}
(i,j) \in \tilde{E} \text{ mit } i < j\\ \iff \exists \text{ Kantenzug } (i,i_1, \dots , i_{l-1},j) \text{ mit }
  i_\nu \in \{1,\dots i-1\}.
\end{multline*}
\item[(ii)] Bei ILU($p$) hat die Matrix $(L\setminus U)$ den
Graphen $\tilde{G} = (V,\tilde{E})$ mit
\begin{multline*}
(i,j) \in \tilde{E} \text{ mit } i < j\\ \iff \exists \text{ Kantenzug der L"ange $p$ } (i,i_1, \dots , i_{l-1},j)\\ \text{ mit }
  i_\nu \in \{1,\dots i-1\}.
\end{multline*}
\end{itemize}
{\em Hinweis:} Verwende Induktion "uber $k$ und formuliere eine geeignete Eigenschaft
f"ur die Zwischenresultate $A^{(k)}$.
\end{aufg}

\subsection{Dynamische ILU}
{\bf Nachteile von ILU(p):} Muster ist "uber ein einfaches Modell, aber ohne
wirkliche R"ucksicht auf numerische Werte festgelegt. Man h"atte aber gerne
$$A = LU - R \text{  mit  } R \text{ "`klein"'}. $$
{\bf Idee:} Lege Muster bei ILU dynamisch fest, d.h w"ahrend der Rechnung.

\medskip

{\bf Typischer Vertreter:} ILUT ("`T"': Threshold = Schwelle )

\begin{defn}
Eine {\em Nullregel} ist eine Vorschrift, bei der Berechnung einer ILU eine
 Zahl (oder Teile einer ganzen Zeile ) auf $0$ zu setzen.
\end{defn}
Damit ergibt sich folgendes Ger"ust f"ur ILU

\begin{alg}[ikj-Form einer allg. ILU] \label{allgemeineILU}
~               % um "Algorithmus" aus dem Kasten rauszubekommen
\vspace*{-2\baselineskip}       % um den Leeraum zu entfernen
\begin{algorithm}
  \begin{algorithmic}[1]
    \FOR{$i = 2, \dots ,n$ }
      \FOR{$ k = 1, \dots ,i-1$}
        \STATE $a_{i,k} = a_{i,k}/a_{k,k}$
        \STATE wende eine Nullregel auf $a_{i,k}$ an
        \IF{$a_{i,k} \not = 0$}
         \FOR{$ j = k+1, \dots , n$}
          \STATE $a_{i,j} = a_{i,j} - a_{i,k} a_{k,j}$
        \ENDFOR
       \ENDIF
      \ENDFOR
    \STATE wende eine Nullregel auf $a_{i,\bullet}$ an \{i-te Zeile\}
    \ENDFOR
  \end{algorithmic}
\end{algorithm}
\end{alg}

\begin{bem}
ILU($p$) passt in das Schema: Die Nullregeln basieren nur darauf, ob $(i,j) \in E$.
\end{bem}

\begin{defn}
ILUT($\tau$,p) verwendet folgende Regeln:
\end{defn}

{\bf F"ur Zeile 4:}
$$
a_{ik} =
\begin{cases}
 0 &\text{falls } |a_{ij}| \leq \tau\\
 \text{unver"andert } & \text{sonst}
\end{cases}
$$

{\bf F"ur Zeile 11:}
\begin{enumerate}
\item $a_{i,j} \leftarrow 0 \text{ , falls} |a_{ij}| \leq \tau$
\item Behalte danach unter den Eintr"agen $a_{i1},\dots,a_{i,i-1}$ ($\simeq L$) nur die $p$ betragsgr"o"sten bei,
      alle anderen werden $0$. Dito mit den Eintr"agen $a_{i,i+1},\dots,a_{i,n}$
    ($\simeq U$)
      ($\Rightarrow \text{max } 2p+1$ Eintr"age der Zeile sind $\neq 0$)
\item[] {\bf Variante f"ur 2.}\\
Es bezeichne $l(i),u(i)$ die Anzahl der Nicht-Nullen in Zeile $i$ von $A$ vor bzw. nach der Diagonalen. Behalte dann
$l(i)+p$ bzw. $u(i) + p$ Eintr"age bei.
\end{enumerate}


\begin{bem}
ILUT($\tau$,p) ist nicht immer eine ILU zu einem Muster $E$, da in Zeile $4$ und Zeile $7$ unterschiedliche
Nullregeln angewendet werden.
\end{bem}
Es gibt nur schwache Aussagen "uber ILUT, typisch ist die folgende.

\begin{defn}
$A \in \rnn$ hei"st $\hat{M}$-Matrix, falls gilt:
\begin{align*}
& a_{ij}  \leq 0 \text{ f"ur } i\neq j,\\
& a_{ii}   > 0  \text{ f"ur } i = 1, \dots n-1, \\
& \forall i < n \hspace{0.3cm} \exists j_i > i \text{ mit }  a_{ij_i} < 0.
\end{align*}
\end{defn}

\begin{sa}
$A$ sei eine diagonal dominante $\hat{M}$-Matrix, d.h.
$$
(a_{i,\bullet}) = \sum_{j=1}^n a_{ij} \geq 0 \text{ f"ur } i = 1,\dots n.
$$
Die Nullregeln in \ref{allgemeineILU} seien beliebig bis auf die Tatsache, dass sie nie auf $a_{i j_i}$ und $a_{ii}$ angewendet werden. 
Dann ist Algorithmus~\ref{allgemeineILU} durchf"uhrbar und es gilt nach Terminierung des Algorithmus:
\begin{align*}
a_{ii} & > 0 \text{ f"ur } i = 1,\dots ,n,\\
a_{nn} & \geq 0, \\
\sum_{j=1}^n a_{ij} & \geq 0 \text{ f"ur } i = 1,\dots ,n. \\
\end{align*}
\end{sa}

{\bf Beweis:}
Wir ben"otigen Bezeichnungen f"ur die Zwischenwerte in Algorithmus~\ref{allgemeineILU}
und formulieren ihn deshalb nochmal.

\begin{alg}[ikj-Form einer allg. ILU mit Zwischenwerten]
~               % um "Algorithmus" aus dem Kasten rauszubekommen
\vspace*{-2\baselineskip}       % um den Leeraum zu entfernen
\begin{algorithm}
  \begin{algorithmic}
    \FOR{$i = 2, \dots ,n$ }
      \FOR{$ k = 1, \dots ,i-1$}
        \STATE $l_{i,k} = a_{i,k}^{(k)}/a_{k,k}^{(k)}$
        \STATE Nullregel
        \FOR{$ j = k+1, \dots , n$}
          \STATE $a_{i,j}^{(k+1)} = a_{i,j}^{(k)} - l_{i,k} a_{k,j}^{(k)}$
        \ENDFOR
      \ENDFOR
    \STATE Nullregel
    \ENDFOR
  \end{algorithmic}
\end{algorithm}
\end{alg}
Wir zeigen, dass f"ur alle $i$ und $k$ gilt:
\begin{align*}
a_{ij}^{(k)} & \leq 0 \text{ f"ur } i\neq j,\\
a_{ii}^{(k)} &  > 0  \text{ f"ur } i < n, \\
\sum_{j=1}^n a_{ij}^{(k)} & \geq 0.
\end{align*}

Der Beweis ist eine Induktion "uber $i$, innerhalb derer eine Induktion
"uber $k$ gemacht wird. Es reicht deshalb, den Schritt von $k$ nach $k+1$
bei festem $i$ anzugeben.

$k = 1$: Hier ist die Behauptung identisch mit den Voraussetzungen des
Satzes.

$k \rightarrow k+1$:
$$
a_{ij}^{(k+1)} =
\begin{cases}
0 & \text{ falls Nullregel zutrifft}\\
a_{i,j}^{(k)} - \underbrace{l_{i,k}}_{\leq 0} \underbrace{a_{k,j}^{(k)}}_{\leq 0} & \text{ sonst, insbesondere} \leq 0
\end{cases} .
$$

F"ur die Zeilensummen gilt weiter
$$\sum_{j=i}^n a_{i,j}^{(k+1)} = \sum_{j=i}^n a_{i,j}^{(k)} - l_{i,k} a_{k,j}^{(k)} - r_{ij}^{(k)}
\overset{(*)}{=} \underbrace{\sum_{j=i}^n a_{i,j}^{(k)}}_{\geq 0} - \underbrace{l_{i,k}}_{\leq 0}
\underbrace{\sum_{j=i}^n a_{k,j}^{(k)}}_{\geq 0} \underbrace{- r_{ij}^{(k)}}_{\geq 0} \geq 0$$
Wobei $(*)$ folgt aus
$$
r_{ij}^{(k)} =
\begin{cases}
0 & \text{ falls Nullregel zutrifft}\\
a_{i,j}^{(k)} - l_{i,k} a_{k,j}^{(k)} & \text{ sonst, insbesondere} \leq 0
\end{cases} .
$$
Zu zeigen bleibt $a_{ii}^{(k+1)} > 0$. Wir wissen 
$$a_{ii}^{(k+1)} \geq \sum_{j=i+1}^n -a_{ij}^{(k+1)} \geq - a_{ij_i}^{(k+1)} = -(a_{ij_i}^{(k)} - l_{i,k} a_{kj_i}^{(k)})
\geq a_{ij_i}^{(k)} > 0.
$$


