\section[Allgemeine Betrachtungen...]{Allgemeine Betrachtungen zur Konvergenz von Krylov-Unterraum-Verfahren}

F"ur jedes Krylov-Unterraum-Verfahren zur L�sung von $Ax=b, A \in \co^{n \times n}$, $b \in \co^n$ gilt
\[
r^{m}=b-Ax^{m}=p_m(A)r^{0}, \quad p_m\in \overline{\Pi}_m.
\]
Ein KUV ist konvergent, wenn
\[
\limm p_m(A)r^{0}=0
\]
gilt. Bei geeigneter Wahl von $p_m$ wird $p_m(A)r^{0}=0$ bereits f"ur ein endliches
$m\le n$; manche Verfahren erreichen dies auch tats"achlich.
Andere Verfahren
(wie z.B. das Richardson-Verfahren) ben"otigen eine unendliche Iterationszahl.

\medskip

Man unterscheidet zwei verschiedene Ans"atze zur Analyse von KUV:
\begin{enumerate}
\item parameterfreie KUV: Sie ben\"otigen keine (detailliertere) Information
\"uber Eigenschaften von $A$ (z.B. das Spektrum). In der Regel bilden sie $p_m(A)$
dann in expliziter Abh"angigkeit
von $r^{0}$ $\hookrightarrow$ sp"ater (vergleiche cg-Verfahren),
\item parameterabh"angige KUV verwenden detailliertere Informationen \"uber $A$, z.B.
\"uber die Lage des Spektrums. In der Regel bilden sie dann $p_m(A)$ unabh"angig von
$r^{0}$. Vorteil: keine Innenproduktbildung,
damit besser parallelisierbar; Nachteil: eventuell weniger effizient.
\end{enumerate}

\textbf{Frage:} Wann gilt $\limm p_m(A)=0\in\cnn$?

W"are $A$ diagonalisierbar, d.h. es gibt eine invertierbare Matrix $T$ und eine Diagonalmatrix
$\Lambda=diag(\lambda_1,...,\lambda_n)$, so dass
\[
A=T\Lambda T^{-1},
\]
dann gilt
\[
p_m(A)=Tp_m(\Lambda)T^{-1}.
\]
Damit ist dann
\[
\limm p_m(A)=0\Leftrightarrow\limm p_m(\lambda)=0 \quad \forall\ \lambda\in\text{spek} (A).
\]
Ist $A$ nicht diagonalisierbar, so werden weitere Bedingungen notwendig.

\begin{sa}\label{KonvergenzKUV_sa}
Es sei
\[
A=TJT^{-1}
\]
mit
\[
J=\left(\begin{array}{ccc}
J_1\\&\ddots\\&&J_k
\end{array}\right)
 \quad
J_l=\left(\begin{array}{cccc}
\lambda_l&1\\
&\ddots&\ddots\\
&&\ddots&1\\
&&&\lambda_l
\end{array}\right) \in \mathbb{C}^{n_l\times n_l}
\]
die Jordan-Normalform von $A$ mit $\lambda_l\in \text{spek}(A) =
\{\lambda_1,...,\lambda_k\}$ (aber Mehrfachnennungen m"oglich). Dann gilt
\[
\limm p_m(A)=0\Leftrightarrow\limm p_m^{(\nu)}(\lambda_l)=0 \quad l=1,...,k,\ \nu=0,...,n_l-1.
\]
\end{sa}
\begin{proof}
Es gilt
\[
p_m(A)=Tp_m(J)T^{-1},
\]
also
\[
\limm p_m(A)=0\Leftrightarrow\limm p_m(J_l)=0\quad l=1,...,k.
\]
Wir betrachten also
\[
J_l=\lambda_l\cdot I_{n_l}+S, \quad S=\left(\begin{array}{cccc}
0&1\\
&\ddots&\ddots\\
&&\ddots&1\\
&&&0
\end{array}\right)\in\mathbb{C}^{n_l\times n_l} \Rightarrow S^{n_l}=0.
\]
Damit ergibt sich (mit $\binom{k}{\nu}=0$, falls $\nu >k$)
\[
J_l^k=\sum\limits_{\nu=0}^{n_l-1}\binom{k}{\nu}S^\nu\lambda_l^{k-\nu}.
\]
Es sei nun $p_m(t)=\sum\limits_{k=0}^{m}c_k t^k$, dann ist
\begin{align*}
p_m(J_l)
&=\sum\limits_{k=0}^{m}c_k \sum\limits_{\nu=0}^{n_l-1}\binom{k}{\nu}S^\nu\lambda_l^{k-\nu}\\
&=\sum\limits_{\nu = 0}^{n_l-1}\frac{1}{\nu!}S^\nu
   \underbrace{ \sum\limits_{k=0}^{m}c_k \frac{k!}{(k-\nu)!}\lambda_l^{k-\nu}}_{p_m^{(\nu)}(\lambda_l)}.
\end{align*}
Hierin bezeichnet $p_m^{(\nu)}$ die $\nu$-te Ableitung des Polynoms $p_m$.

In der Matrix $S^\nu$ ist genau die $\nu$-te obere Nebendiagonale (mit Einsen)
besetzt.  Es gilt also
\[
\limm p_m(J_l)=0\Leftrightarrow \limm p_m^{(\nu)}(\lambda_l)=0 \quad \nu=0,...,n_l-1.
\]
\end{proof}

\begin{cor}\label{Konvergenzdiag_kor}
Sei $A\in\cnn$ diagonalisierbar (z.B. hermitesch oder normal\footnote{d.h. $AA^*=A^*A$, wobei $A^*$ die Adjungierte von $A$ bezeichnet}). Dann gilt
\[
\limm p_m(A)=0\Leftrightarrow\limm p_m(\lambda)=0 \quad \forall\ \lambda\in \text{spek}(A).
\]
\end{cor}

F"ur diagonalisierbares $A$ ist also die
Gr"o"se $\underset{\lambda\in\text{spek}(A)}{\max}|p_m(\lambda)|$ ein 
Ma"s f"ur die Gr"o"se von $p_m(A)$, sie ist
sogar eine Norm von $p_m(A)$:
\[
A=SDS^{-1}, \quad D=\diag(\lambda_1,...,\lambda_n). 
\]
Nehme
\[
\|x\|_S=\|S^{-1}x\|_2,
\]
dann ist
\begin{align*}
\|A\|_S=\underset{\|x\|_S\ne 0}{\max} \frac{\|Ax\|_S}{\|x\|_S}
&=\underset{\|x\|_S\ne 0}{\max}\frac{\|DS^{-1}x\|_2}{\|S^{-1}x\|_2}\\
&=\underset{y\ne 0}{\max}\frac{\|Dy\|_2}{\|y\|_2}=\|D\|_2\\
&=\underset{\lambda\in\text{spek}(A)}{\max}|\lambda|
\end{align*}
und analog
\[
\|p_m(A)\|_S=\underset{\lambda\in\text{spek}(A)}{\max}|p_m(\lambda)|.
\]
Minimierung von
\[
\underset{\lambda\in\text{spek}(A)}{\max}|p_m(\lambda)|
\]
minimiert also auch $\|p_m(A)\|_S$.

\medskip

\textbf{Bemerkung:} $\|\cdot\|_S$ ist i.A. eine "`schiefe"' Norm. Ist $A$
jedoch normal, so kann man $S$ unit"ar w"ahlen, so dass $\|x\|_S = \|x\|_2$
und damit $\|A\|_S = \|A\|_2 =
\underset{\lambda\in\text{spek}(A)}{\max}|p_m(\lambda)|$.


\bigskip

Im diagonalisierbaren Fall l"ost damit ein bestm"ogliches Iterationsverfahren
f"ur jedes $m$ die MinMax-Aufgabe
\begin{equation} \label{minimax_eq}
\underset{p_m\in\overline{\Pi}_m}{\min} \  \underset{\lambda\in\text{spek}(A)}{\max}
|p_m(\lambda)|.
\end{equation}


Dies ist jedoch in den meisten F"allen praktisch nicht m"oglich, schon alleine
weil man i.A. die Eigenwerte von $A$ nicht alle kennt.

Parameterabh"angige KUV verwenden deshalb nur partielle Informationen "uber
$\text{spek}(A)$, um mittels dieser 
Informationen m"oglichst "`gute"' Polynome $p_m(t)$ zu finden.

\medskip

Es sei z.B. (mittels Satz von Gerschgorin o."A.) bekannt, dass
\[
\text{spek}(A)\subset D(\zeta,\rho)=\{z\in\co:\ |z-\zeta|\leq\rho\}.
\]

Wie lautet die L"osung von 
\[
\underset{p_m\in\overline{\Pi}_m}{\min} \  \underset{\lambda\in D(\zeta,\rho)}{\max}|p_m(\lambda)|\;?
\]

\textbf{Beachte:} $0 \notin D(\zeta,\rho)$ ist hier eine vern"unftige
zus"atzliche Bedingung, da 
ansonsten die L"osung $p_m\equiv1$ lautet f"ur alle $m$, d.h.\ es
resultiert ein nicht konvergentes KUV.

\begin{sa}\label{Minmaxkreis_sa}
Sei $\rho>0$, sei $\gamma\in\co$ mit $|\gamma|>\rho$. Dann l"ost das Polynom
\[
p_m(z)=\frac{z^m}{\gamma^m}
\]
die Aufgabe
\[
\underset{p_m\in\Pi_m,\ p_m(\gamma)=1}{\min} \  \underset{\lambda\in D(0,\rho)}{\max}|p_m(\lambda)|,
\]
der Wert ist $|\rho/\gamma|^m$.
\end{sa}
\textbf{Bemerkung:} $|\rho/\gamma|$ ist klein, wenn $\gamma$ weit vom Kreis
$D(0,\rho)$ entfernt ist. 

Zum Beweis des Satzes ben"otigen wir den Satz von Rouch\'e.

\begin{lem}[Rouch\'e]\label{Rouche_lem}
Es seien $f,g$ holomorph im Gebiet $\Omega\subset \co$. Es sei $\Gamma$ ein
einfach geschlossener, in $\Omega$
nullhomologer Weg (z.B. der Rand eines Kreises in $\co$), so dass gilt
\[
|f(\zeta)-g(\zeta)|<|g(\zeta)| \quad \text{f"ur alle } \zeta\in\Gamma.
\]
Dann haben $f$ und $g$ gleich viele Nullstellen im Inneren von $\Gamma$.
\end{lem}
Vergleiche \textsc{Remmert}, Funktionentheorie I. Grundlehren, Springer,
4. Auflage; Seite 310f.


\begin{proof}
Sei $p_m^*(z)=\frac{z^m}{\gamma^m}$. Sei $p_m\in \Pi_m,\ p_m(\gamma)=1$ ein Polynom mit
\[
\underset{\lambda\in D(0,\rho)}{\max}|p_m(\lambda)|<\underset{\lambda\in D(0,\rho)}{\max}|p_m^*(\lambda)|.
\]
Dann gilt  
\[
|(p_m^*(z)-p_m(z))-p_m^*(z)|=|p_m(z)|<\left|\frac{\rho}{\gamma} \right|^m=|p_m^*(z)| \enspace \mbox{f"ur } z \in \partial D(0,\rho).
\]
Also besitzen $p_m^*-p_m$ und $p_m^*$ gleich viele Nullstellen im Inneren von
$D(0,\rho)$, n"amlich $m$ St"uck. Au"serdem
gilt $(p_m^*-p_m)(\gamma)=1-1=0$, also besitzt das Polynom $p:=p_m^*-p_m\in\Pi_m$ mindestens $m+1$ Nullstellen,
ist also nach dem Fundamentalsatz der Algebra das Nullpolynom. 
\end{proof}

\medskip

\begin{aufg} Finde ausgehend von Satz \nref{Minmaxkreis_sa} die L"osung von
\[
\underset{p_m\in \overline{\Pi}_m,}{\min}\ \underset{\lambda\in D(1,\rho)}{\max}|p_m(\lambda)|.
\]
\end{aufg}

\textbf{Antwort: s. "Ubung.}
\medskip

Wir erhalten so die folgende Interpretation: 
Bez"uglich der Information
\[
\spek(A) \subseteq D(1,\rho)
\]
mit $\rho < 1$ ist das Richardson-Verfahren \emph{optimal}.
