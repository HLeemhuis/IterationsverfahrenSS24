
\section{Krylov-Unterr�ume}

\begin{defn}\label{KU_def}
Zu $r\in\mathbb{C}^n\setminus\{0\}$ und $A\in\cnn$ definieren wir den \emph{Krylov-Unterraum} der
Stufe $m$ als
\begin{align*}\index{Krylov-Unterraum}
K_m(A,r)&:=\text{span}\{r,Ar,A^2r,...,A^{m-1}r\}\\
&=\{y\in\mathbb{C}^n:\ y=\sum\limits_{j=0}^{m-1}\alpha_j A^j r\}\\
&=\{y\in\mathbb{C}^n:\ y=p_{m-1}(A)r, \quad p_{m-1}\in \Pi_{m-1}\},
\end{align*}
wobei $\Pi_{m-1}$ der Vektorraum der Polynome vom H�chstgrad $m-1$ ist, d.h.
\[\Pi_{m-1}:=\left\{p:\ p=\sum\limits_{j=0}^{m-1}a_jx^j\right \}.\]
\end{defn}

Krylov-Unterr�ume sind geschachtelt. Da sie alle Teilr�ume von $\mathbb{C}^n$ sind,
werden sie irgendwann nicht mehr gr��er werden. Wir untersuchen dies genauer. 

\begin{lem}\label{min_lem}
Zu $r\in\mathbb{C}^n$ existiert ein minimales $m^*\in\mathbb{N}_0$, $m^*\le n$ und ein Polynom
$p_{m^*}\ne 0$ vom Grad $m^*$ mit
\begin{enumerate}
\item $p_{m^*}(A)r=0$ (dieses Polynom ist bis auf skalare Vielfache eindeutig;
alle Polynome $p$ mit $p(A)r = 0$ bilden ein Ideal).
\item Es gilt
\begin{align*}
\dim(K_{m}(A,r))&\le \dim(K_{m+1}(A,r)), \quad  m=1,\ldots,m^*,\\
\dim(K_{\hat m}(A,r))&=\hat m, \quad \hat m=1,...,m^*,\\
K_{m^*}(A,r) &=K_{m^*+1}(A,r)=K_{m^*+2}(A,r)=\ldots .
\end{align*}
\end{enumerate}
\end{lem}
\begin{proof}
\begin{enumerate}
\item Das Minimalpolynom $p_A$ von $A$  erf�llt $p_A(A)r=0$, also existiert auch ein Polynom minimalen Grades
mit der geforderten Eigenschaft.
\item Es ist klar, dass
\[\dim(K_{m}(A,r))\le \dim(K_{m+1}(A,r))\le \dim(K_{m}(A,r))+1\]
gilt, da
\[K_{m+1}(A,r)=K_{m}(A,r)+\left\langle A^mr\right\rangle .\]
\begin{itemize}
\item F�r
\[
p_{m^*}(t)=\sum\limits_{j=0}^{m^*}c_jt^j, \quad c_{m^*}\ne0
\]
folgt sofort aus $p_{m^*}(A)r=0$
\[
A^{m^*}r\in K_{m^* }(A,r)\Rightarrow K_{m^*}(A,r) = K_{m^*+1}(A,r).
\]
Jedes Polynom $p$ mit $\deg(p) = m > m^* $ besitzt eine Darstellung
$p = q \cdot p_{m^*} + s$ mit $\deg(s) < m^*$. Also ist $p(A)r = s(A)r$ und
damit $K_m(A,r) \subseteq K_{m^*}(A,r)$, also $K_m(A,r) = K_{m^*}(A,r)$.

\item Es sei $\widetilde{m}$ der erste Index mit $K_{\widetilde{m}}(A,r)=K_{\widetilde{m}+1}(A,r)$, dann ist
\begin{align*}
A^{\widetilde{m}}r\in K_{\widetilde{m}}
	&\Leftrightarrow\exists\text{ Polynom $p$ vom Grad $\widetilde{m}$ mit }\widetilde{p}(A)r=0\\
	&\Longrightarrow\widetilde{m}\ge m^*\Rightarrow\widetilde{m}=m^*.
\end{align*}
\end{itemize}
\end{enumerate}
\end{proof}

\begin{sa}\label{loes_sa}
Sei $A\in\cnn$ regul�r und $x^{0}\in\mathbb{C}^n,\ r^{0}=b-Ax^{0}$. Dann erf�llt die L�sung $x^*=A^{-1}b$ von
\[
Ax=b
\]
die Beziehung
\begin{align*}
x^*&\in x^{0}+K_{m^*}(A,r^{0}),\\
x^*&\notin x^{0}+K_{m}(A,r^{0}), \quad m<m^*. 
\end{align*}
\end{sa}
\begin{proof}
Sei $p_{m^*}(t)=\sum\limits_{j=0}^{m^*}c_jt^j$ wie in Lemma \nref{min_lem}.
Dann ist $c_0\ne 0$, denn sonst g�lte
\[
\sum\limits_{j=1}^{m^*}c_jA^jr=0
	\Leftrightarrow A\underbrace{\left(\sum\limits_{j=1}^{m^*}c_jA^{j-1}r \right)}_{\ne 0}=0,
\]
im Widerspruch zur Regularit�t von $A$.

\medskip

Es gilt also
\begin{alignat*}{3}
p_{m^*}(A)r^{0}=0
	&\Leftrightarrow A^{-1}p_{m^*}(A)r^{0}=0\\
	&\Leftrightarrow A^{-1}\sum\limits_{j=0}^{m^*}c_jA^jr^{0}=0\\
	&\Leftrightarrow A^{-1}\sum\limits_{j=0}^{m^*}c_jA^j(b-Ax^{0})=0\\
	&\Longrightarrow x^*=A^{-1}b=x^{0}+\dfrac{1}{c_0}\sum\limits_{j=1}^{m^*}c_jA^{j-1}r^{0}\\
	&  \phantom{\Longrightarrow x^* }\    \in x^{0}+K_{m^*}(A,r^{0}).
\end{alignat*}
Die Annahme $x^*\in x^{0}+K_{m^*-1}(A,r^{0})$ f�hrt auf ein Polynom $p\ne 0$ mit $\deg(p)<m^*$ und
$p(A)r^{0}=0$, im Widerspruch zur Minimalit�t von $m^*$.
\end{proof}

\begin{defn}\label{KUV_def}
Ein \emph{Krylov-Unterraum-Verfahren} (KUV) zur L�sung von $Ax=b$ ist ein Unterraumverfahren mit Startwert $x^{0}$,
Startresiduum $r^{0}=b-Ax^{0}$ und
\[
x^{m}\in x^{0}+K_m(A,r^{0}).
\]
Es ist also $x^{m}=x^{0}+q_{m-1}(A)r^{0},\ \deg(q_{m-1})\le m-1$ und
\begin{align*}
r^{m}=b-Ax^{m}&=r^{0}-Aq_{m-1}(A)r^{0}\\
&=p_m(A)r^{0}, \quad p_m\in \overline{\Pi}_{m}=\{p\in \Pi_{m}:\ p(0)=1\}\\
& \quad  \in K_{m+1}(A,r^{0}).
\end{align*}
$p_m$ und $q_{m-1}$ (mit $p_m(t)=1-t\cdot q_{m-1}(t)$) hei�en auch die zum KUV
geh�rigen \emph{Verfahrenspolynome}.
\end{defn}

Umgekehrt definiert jede Folge von Polynomen $q_{m-1} \in \Pi_{m-1}$ ein
KUV mit $x^{m} = x^{0} + q_{m-1}(A)r^{0}$ und ebenso jede Folge von Polynomen
$p_m \in \overline{\Pi}_{m}$  (mit $r^{m} = p_m(A)r^{0}$). Grunds�tzlich
besteht immer der Zusammenhang
\begin{equation} \label{pqbeziehung_eq}
p_m(t) = 1 - tq_{m-1}(t).
\end{equation}

Wir interpretieren nun einfache bekannte Verfahren als KUV.

\medskip

\textbf{Erinnerung:} Es sei $A=(a_{i,j})\in\cnn$, dann sind $D,L,U \in \mathbb{C}^{n \times n}$
gegeben durch
\begin{align*}
D&=\left(\begin{array}{ccccc}
a_{1,1}\\
&\ddots\\
\phantom{-a_{2,1}}&&\ddots&\raisebox{12pt}[-12pt]{\Huge 0}\\
&\raisebox{-12pt}[12pt]{\Huge 0}&&\ddots\\
\phantom{-a_{2,1}}&\phantom{-a_{2,1}}&\phantom{\vdots}&\phantom{\vdots}&a_{n,n}
\end{array}\right)&&\text{(Diagonalteil)}\\\\
L&=\left(\begin{array}{ccccc}
0\\
-a_{2,1}&\ddots\\
\vdots&&\ddots&\raisebox{12pt}[-12pt]{\Huge 0}\\
\vdots&&&\ddots\\
-a_{n,1}&\hdots&\hdots&-a_{n,n-1}&0
\end{array}\right)&&\text{\parbox{5cm}{(negativer)\\\hfill linker unterer Dreiecksteil}}\\\\
U&=\left(\begin{array}{ccccc}
0&-a_{1,2}&\hdots&\hdots&-a_{1,n}\\
&\ddots&&&\vdots\\
&&\ddots&&\vdots\\
&\raisebox{-12pt}[12pt]{\Huge 0}&&\ddots&-a_{n-1,n}\\
&&&&0
\end{array}\right)&&\text{\parbox{5cm}{(negativer)\\\hfill rechter oberer Dreiecksteil}}
\end{align*}

Es ist also $A = D-L-U$.

Wir betrachten nun die drei Standard-Verfahren:

\medskip

\begin{tabular}{ll}
Jacobi:&$x^{m+1}=D^{-1}((L+U)x^{m}+b)$\\
Gau�-Seidel:&$x^{m+1}=(D-L)^{-1}(Ux^{m}+b)$\\
SOR:&$x^{m+1}=(\frac{1}{\omega}D-L)^{-1}((\frac{1-\omega}{\omega}D+U)x^{m}+b), \quad \omega\in\mathbb{R}\setminus\{0\} $
\end{tabular}

\medskip

und interpretieren sie folgenderma�en als KUV:

\begin{description}
\item[Jacobi:] Setze $H:=D^{-1}(L+U)$ und $A' = D^{-1}A, \; b' = D^{-1}b$. Mit $r^{0}=b'-A'x^{0}$ erhalten wir
\begin{align*}
x^{m+1}&=Hx^{m}+b'\\
r^{m+1}&=b'-A'x^{m+1}=b'-A'(Hx^{m}+b')\\
&=b'-A'((I-A')x^{m}+b')\\
&=b'-A'(x^{m}+r^{m})\\
&=r^{m}-A'r^{m}=(I-A')r^{m}\\
\Rightarrow r^{m}&=(I-A')^mr^{0}.
\end{align*}
Also ist das Jacobi-Verfahren ein KUV bez�glich $A'x=b'$ mit
\begin{align*}
p_m(t)&=(1-t)^m,\\
q_{m-1}(t)&=\dfrac{1-p_m(t)}{t}=\sum\limits_{j=0}^{m-1}(-1)^{j}\binom{m}{j+1}t^j.
\end{align*}
\item[Gau�-Seidel:] ist KUV bez�glich $\underbrace{(D-L)^{-1}A}_{=A'}x=\underbrace{(D-L)^{-1}b}_{b'}$ mit
\begin{align*}
p_m(t)&=(1-t)^m, \\
q_{m-1}(t)&=\dfrac{1-p_m(t)}{t}=\sum\limits_{j=0}^{m-1}(-1)^{j}\binom{m}{j+1}t^j.
\end{align*}
\item[SOR:] ist KUV bez�glich $\underbrace{\left(\textstyle\frac{1}{\omega}D-L \right )^{-1}A}_{=A'}x
					=\underbrace{ \left (\textstyle\frac{1}{\omega}D-L \right )^{-1}b}_{b'}$ mit
\begin{align*}
p_m(t)&=(1-t)^m,\\
q_{m-1}(t)&=\dfrac{1-p_m(t)}{t}=\sum\limits_{j=0}^{m-1}(-1)^{j}\binom{m}{j+1}t^j.
\end{align*}
\end{description}

Allgemein gilt folgender

\begin{sa}\label{KUV_sa}
Ein Iterationsverfahren
\begin{equation}
x^{m+1}=Hx^{m}+b' \label{iter_eq}
\end{equation}
zur L�sung von $Ax=b$ mit $A=M-N$, $H=M^{-1}N,\ b'=M^{-1}b$ ist ein KUV (bez�glich des
Systems $A'x=b', \; A' = M^{-1}A$) mit $p_m(t)=(1-t)^m$.
\end{sa}

\textbf{Bemerkung:} Die Verfahrenspolynome h�ngen bei diesen einfachen Iterationen
nicht vom Startresiduum ab.

\begin{defn}\label{S16}
Ein KUV zur L�sung von $Ax=b$ mit regul�rem $A$ hei�t \emph{konvergent}, falls 
\[
\limk r^{k}=0\Leftrightarrow \limk x^{k}=x^*=A^{-1}b.
\]
\end{defn}

\begin{sa}\label{konvergent_sa}
Ein Iterationsverfahren der Gestalt \eqref{iter_eq} ist konvergent, falls 
\[
\rho(I-M^{-1}A)<1
\]
gilt.
\end{sa}
\begin{proof}
Es gilt $r^{m}=(I-M^{-1}A)^mr^{0}$, wegen $\rho(I-M^{-1}A)<1$ gilt
$\limk (I-M^{-1}A)=0$, also $\limk r^{k}=0$.
\end{proof}

\begin{defn}\label{Richardson1_def}
Das \emph{Richardson-Verfahren erster Ordnung} zur L�sung von $Ax=b$ ist die Iteration
\[
x^{m+1}=(I-A)x^{m}+b=x^{m}+r^{m}.
\]
\end{defn}

Die Richardson-Iteration erster Ordnung ist im Fall $D = I$ mit der Jacobi-Iteration identisch. F�r die Verfahrenspolynome gilt wieder
\[
   p_m(t) = (1-t)^m.
\]

\begin{cor} \label{RichardsonKonvergenz_sa}
Das Richardson-Verfahren erster Ordnung zur L�sung von $Ax=b$  konvergiert, 
falls
\[
\rho(I-A)<1,
\]
z.B. falls $\text{spek}(A)\subset (0,2)$.
\end{cor}

\begin{defn}\label{relRichardson_def}
Das \emph{relaxierte Richardson-Verfahren erster Ordnung} zur L�sung von $Ax=b$ ist die Iteration
\[
x^{m+1}=x^{m}+ \alpha r^{m}.
\]
Hierbei ist $\alpha \in \co$ ein fester Parameter, der bei geeigneter Wahl die Konvergenz beschleunigt.
\end{defn}

Die Verfahrenspolynome f�r das relaxierte Richardson-Verfahren sind
\begin{equation}\label{prelRichardson_def}
   p_m(t) = (1-\alpha t)^m.
\end{equation}


\begin{defn}\label{Richardson2_def}
Das \emph{Richardson-Verfahren zweiter Ordnung} zur L�sung von $Ax=b$ ist bei gegebenem $x^{0}$ die Iteration
\begin{align*}
x^{1}&=(I-A)x^{0}+b,\\
x^{m+1}&=\alpha((I-A)x^{m}+b)+(1-\alpha)x^{m-1},\quad m\ge 1,
\end{align*}
mit $\alpha\in \co$ fest.
\end{defn}

Interpretation als KUV:
\begin{align*}
r^{1}&=(I-A)r^{0} \quad \Rightarrow \quad p_1(t)= 1-t\\
r^{m+1}&=b-Ax^{m+1}\\
&=b-\alpha A((I-A)x^{m}+b)-(1-\alpha)Ax^{m-1}\\
&=\alpha r^{m}-\alpha Ar^{m}+(1-\alpha)r^{m-1}\\
&=\alpha(I-A)r^{m}+(1-\alpha)r^{m-1}\\
\intertext{also}
p_{m+1}(t)&=\alpha(1-t)p_m(t)+(1-\alpha)p_{m-1}(t), \quad m\geq 1.
\end{align*}

Solche \emph{3-Term-Rekursionen} sind typisch f�r "`fortgeschrittene"' KUV.
F�r eine sp�tere Analyse halten wir fest, dass gilt
\begin{equation} \label{matrixrekursion2Richardson_eq}
\left(\begin{array}{l}
p_{m+1}(t)\\
p_m(t)
\end{array}\right)
=
\underbrace{\left(\begin{array}{cc}
\alpha(1-t)&1-\alpha\\ 1&0
\end{array}\right)}_{B}
\cdot
\left(\begin{array}{l}
p_m(t)\\
p_{m-1}(t)
\end{array}\right).
\end{equation}
