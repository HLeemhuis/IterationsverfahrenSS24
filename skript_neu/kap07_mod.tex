\section[KUV mit minimalem Residuum]{Krylov-Unterraum-Verfahren mit minimalem Residuum}

{\bf Gegeben: }
\begin{quote}
$A \in \cnn$ mit $A \neq A^H$,  $A$ regul\"ar.
\end{quote}
{\bf Ziel: }
\begin{quote}
L\"ose $Ax = b$ mit Startvektor $x^0$ und $r^0 = b-Ax^0$.
\end{quote}
{\bf Idee:}
\begin{quote}
Bestimme ONB von $K_m(A,r^0)$ mit Arnoldi-Prozess und errechne $x^m \in x^0 + K_m(A,r^0)$, so dass
\end{quote}
\[
\|b-Ax^m\|_2 = \underset{x \in x^0 + K_m(A,r^0)}{\min} \|b-Ax\|_2.
\]

{\bf Ansatz:}
\begin{quote}
$x = x^0 + V_m z^m$ mit $z^m = \left(z_1,\cdots,z_m\right)^T \in \co^m$ sowie
$V_m = \left[ v^1,\cdots,v^m \right]$.
\end{quote}

Damit erhalten wir
\[
b - Ax = r^0 - AV_m z^m = r^0 - V_{m+1}H_{m+1,m}z^m.
\]

Mit $r^0 = h_{1,0} v^1$ und $h_{1,0} = \|r^0\|$ gilt nun
\[
b-Ax = V_{m+1} \left[
\left(%
\begin{array}{c}
  h_{1,0} \\
  0 \\
  \vdots \\
  0 \\
\end{array}%
\right) - H_{m+1,m} z^m \right]
\]
und somit
\[
\|b-Ax\|_2 = \|h_{1,0}e^1 - H_{m+1,m}z^m\|_2.
\]
Hierbei ist $e^1$ der erste Einheitsvektor in $\co^{m+1}$,
$H_{m+1,m} \in \mathbb{C}^{(m+1)\times m}$
aber $z \in \mathbb{C}^m$.

\clearpage

{\bf Exkurs:}

\begin{quote}
Zu l\"osen ist also ein "uberbestimmtes Least Squares Problem $Bx = c$ mit
$B\in \mathbb{C}^{k \times l}, \; k > l$. Gesucht ist
ein $x$, welches $\| Bx-c \|_2^2$ minimiert. Dies findet man unter Verwendung der
QR-Faktorisierung.

{\bf Ansatz:}
\begin{quote}
Faktorisiere $ B = QR$ mittels QR-Zerlegung. Dann folgt
\end{quote}
\begin{align*}
\|Bx-c\|_2^2 &= \|Q^H(Bx-c)\|_2^2\\
&= \|Rx - Q^Hc\|_2^2.
\end{align*}
Das Minimum wird erreicht f\"ur
\[
x = R_1^{-1}( (Q^Hc)(1:l) ),
\]
wobei
\[
R = \left(\begin{array}{ccc} & R_1 & \\
    0 & \cdots & 0 \end{array}
    \right), \enspace R_1 \in \co^{l\times l} \mbox{ obere Dreiecksmatrix }.
\]
\end{quote}

In unserem Falle ist $H_{m+1,m}$ eine obere Hessenberg Matrix. Ihre QR-Fak\-to\-ri\-sierung
kann mittels Jacobi Rotationen bestimmt werden. Es kann sogar $Q$ in der
QR-Faktorisierung von $H_{m+1,m}$ auf die f"ur $H_{m+2,m+1}$ aufdatiert werden.

\bigskip

{\bf Notation:}
\[
Q_{m+1} H_{m+1,m} = R_{m+1,m} \enspace \mbox{QR-Faktorisierung}
\]
mit
\[
Q_{m+1} = J_m^{(m+1)}\cdots J_1^{(m+1)},
\]
wobei $J_j^{(m+1)}$ eine Jacobi Rotation in $\co^{m+1}$ auf den Zeilen $j$
und $j+1$ ist, welche das `untere Nebendiagonalelement' $h_{j+1,j}$ eliminiert,
\[
J_j^{(m+1)} = \left(%
\begin{array}{cccc}
  I &  &  &  \\
   & \bar{c}_j & \bar{s}_j &  \\
   & -s_j & c_j &  \\
   &  &  & I \\
\end{array}%
\right).
\]
Um anzugeben, wie man $J_m^{(m+1)}$ bestimmt, beobachten wir zun\"achst
\begin{eqnarray*}
J_{m-1}^{(m+1)} \cdots J_1^{(m+1)} H_{m+1,m} &=& J_{m-1}^{(m+1)} \cdots J_1^{(m+1)}
    \left(\begin{array}{c|c} 
  H_{m,m-1}& \begin{array}{c}
           h_{1,m} \\
  \vdots \\
h_{m,m}
  \end{array}\\
\hline
  0         & h_{m+1,m}
  \end{array}
 \right) \\
&=&
    \left( \begin{array}{cc}
  R_{m,m-1} & \eta^{(m)} \\
  0         & h_{m+1,m}
  \end{array}
 \right)
\end{eqnarray*}
mit
\[
\eta^{(m)} = (r_{1,m},\ldots,r_{m-1,m},\eta_{m,m})^T,
\]
wobei
\begin{eqnarray*}
r_{1,m} &=& \bar{c}_1h_{1,m} + \bar{s}_1h_{2,m}, \\
\eta_{1,m} &=& -s_1h_{1,m} + c_1h_{2,m}, \\
r_{2,m} &=& \bar{c}_2\eta_{2,m} + \bar{s}_2h_{3,m}, \\
\eta_{2,m} &=& -s_2\eta_{1,m} + c_2h_{3,m}, \\
& \vdots & \\
\eta_{m,m} &=& -s_{m-1}\eta_{m-1,m} + c_{m-1}h_{m,m} .
\end{eqnarray*}
Damit erhalten wir f"ur die Parameter $c_m,s_m$ der Jacobi Rotation
$J^{(m+1)}_m$, welche $(0,\ldots,0,\eta_{m,m},h_{m+1,m})^T$ abbildet auf
$(0,\ldots,0,r_{m,m},0)^T$ die Werte 
\begin{equation} \label{rotparam_eq}
  c_m = \eta_{m,m}/\sqrt{|\eta_{m,m}|^2 + |h_{m+1,m}|^2}, \
s_m = h_{m+1,m}/\sqrt{|\eta_{m,m}|^2 + |h_{m+1,m}|^2}  \hspace*{-0.6cm}
\end{equation}
und
\[
r_{m,m} = \bar{c}_m\eta_{m,m} + \bar{s}_mh_{m+1,m} = \sqrt{|\eta_{m,m}|^2 + |h_{m+1,m}|^2} .
\]

% Vorlesung vom 20.05.03

Die L"osung des LS-Problems
\[\min \|h_{1,0} e^1 - H_{m+1,m} z^m \|_2 \]
lautet damit
$$ z^m = h_{1,0} R_m^{-1} ((Q_{m+1} e^1)(1:m)), \enspace \mbox{ wobei }
 R_{m+1,m} = \left( \begin{array}{c} R_m \\
                    0 \cdots 0
                    \end{array}
             \right),
$$
und der Wert des Minimums ist
$$ h_{1,0} ( Q_{m+1} e^1)_{m+1}. $$
Dieser Wert kann ohne Kenntnis von $z^m$ wie folgt bestimmt werden:
Setze 
$$ Q_m e^1=h_{1,0} \left(\begin{array}{c}
                        \epsilon_1\\
                        \vdots\\
                        \epsilon_{m-1} \\
                        \tilde{\epsilon}_{m}
               \end{array}\right). $$
Dann ist
\[
Q_{m+1}e^1 = J_m^{(m+1)} \left( \begin{array}{cc} Q_m & 0 \\
                                               0 & 1 
                                  \end{array}
                         \right) e^1
 = J_m^{(m+1)} \cdot
      \left(\begin{array}{c}
                        \epsilon_1\\
                        \vdots\\
                        \epsilon_{m-1} \\
                        \tilde{\epsilon}_{m} \\
                          0
               \end{array}\right)
 =
\left(\begin{array}{c}
                        \epsilon_1\\
                        \vdots\\
                        \epsilon_{m} \\
                        \tilde{\epsilon}_{m+1}
               \end{array}\right),
\]
mit
\begin{align*}
  \epsilon_m & =  \tilde{\epsilon}_m\overline{c}_m, \\
  \tilde{\epsilon}_{m+1} & =  \tilde{\epsilon}_m\overline{s}_m. \\
\end{align*}
Insbesondere gilt
$$ \tilde{\epsilon}_{m+1} = \prod \limits_{j=1}^{m} \overline{s}_j ,$$
so dass wir f"ur den Wert des Minimums den Ausdruck
\[
  h_{1,0}\tilde{\epsilon}_{m+1} = \|r^0\| \cdot \tilde{\epsilon}_{m+1}
\]
erhalten.

Wir haben jetzt alle Zutaten f"ur das Verfahren
GMRES (Generalized Minimal Residual):

\clearpage


\begin{alg}[GMRES (Saad, Schultz, 1986)] \label{GMRES_alg}
~               % um "3.3 Algorithmus" aus dem Kasten rauszubekommen
\vspace*{-2\baselineskip}       % um den Leeraum zu entfernen
\begin{algorithm}
  \begin{algorithmic}
    \STATE w\"ahle $x^0$, setze $r^0 = b-Ax^0,\ \rho_0=h_{1,0}=\|r^0\|_2 $
    \FOR{$m = 1,2, \dots$, bis $\rho_m$ klein genug}
      \STATE bestimme n\"achsten Vektor $v^m$ und Koeffizienten
    $h_1^m, \ldots ,h_{m+1}^m$ aus Arnoldi-Prozess
      \STATE bestimme $s_m,c_m$ nach \eqnref{rotparam_eq}
      \STATE datiere $\epsilon_m$ und $\tilde{\epsilon}_{m+1}$ auf
      \STATE setze $\rho^m = \rho_{m-1}\cdot \overline{s}_m $
             \enspace \COMMENT{ $\rho_m = \|r^m\|$} 
      \IF{$\rho_m$ klein genug}
          \STATE l\"ose $R_{m} z^{m} =  \left( \epsilon_1, \ldots, \epsilon_m \right)^T$
          \STATE setze $x^{m} = x^0 + V_{m} z^{m}$
      \ENDIF
    \ENDFOR
  \end{algorithmic}
\end{algorithm}
\end{alg}

\begin{bem}
 Speicher- und Rechenaufwand (ohne MVMs) verh\"alt sich wie $\mathcal{O}(mn + m^2)$
 $\curvearrowright$ Verfahren wird schnell impraktikabel.
\end{bem}
Es bezeichne $x^m$ die GMRES-Iterierte nach jedem Schritt, also die L\"osung
von $$\min\limits_{x \in x^0+K_m(A,r^0)} \|b-Ax\|_2. $$
Dann gilt nat"urlich $$ \|r^0\|\geq\|r^1\|\geq \ldots .$$
Das folgende Beispiel zeigt, dass hier lange Gleichheit herrschen kann.

\begin{bsp} \label{stagnation_bsp}
Sei
\[
 A = \left(\begin{array}{cccc}
                        0 &        & & 1\\
                        1 & \ddots & & 0\\
                          & \ddots & \ddots & \vdots \\
                          & & 1 & 0
               \end{array}\right),\enspace x^0=0,\;  b=e^1=r^0.
\]
Dann ist $K_m(A,r^0)=  \spann\{e^1,\ldots,e^m\}$, $e^i$ = $i$-ter Einheitsvektor.
L\"osung von  
$$\min\limits_{x \in K_m(A,r^0)} \|e^1-Ax\|_2, $$
ist f"ur $m=1,2,\ldots,n-1$ wegen
$Ax\in  \spann\{e^2,\ldots,e^{m+1}\}$ stets $x^m=0$.
Es gilt also f\"ur die GMRES-Iterierten
$$ x^0=x^1=\ldots=x^{n-1}=0,\; x^n=e^n = A^{-1}b $$ und 
$$ r^0=r^1=\ldots=r^{n-1}=e^1, $$
also
$$\|r^0\|=\ldots=\|r^{n-1}\|=1. $$
Beachte: $\spek(A)=\{ e^{\frac{2\pi ik}{n}} , \;  k=0,\ldots,n-1\} $. Ein so
um die 0 verteiltes Spektrum ist f"ur KUV prinzipiell schwierig, weil die
Verfahrenspolynome ja $p(0) = 1$ erf"ullen m"ussen.
\end{bsp}

Idee f\"ur "`praktikable"' GMRES-Variante: "`Restarted"'GMRES=GMRES($k$)
mit restart Wert $k$.
\begin{alg}[GMRES($k$)]
~               % um "3.3 Algorithmus" aus dem Kasten rauszubekommen
\vspace*{-2\baselineskip}       % um den Leeraum zu entfernen
\begin{algorithm}
  \begin{algorithmic}
    \STATE w\"ahle $x^0$
    \FOR{$m = 1,2,\dots$}
      \STATE f\"uhre $k$ Schritte von GMRES aus mit Startvektor $x^{m-1}$
      \STATE Ergebnis ist $x^m$
    \ENDFOR
  \end{algorithmic}
\end{algorithm}
\end{alg}
Es ist nun $$ x^m \in x^{m-1}+K_k(A,r^{m-1}), r^{m-1}=b-Ax^{m-1},$$
aber es gilt i.A. {\em nicht}
\[
x^m \in x^0 + K_{km}(A,r^0).
\]

