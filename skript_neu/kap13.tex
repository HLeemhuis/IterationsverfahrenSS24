\section[Modellproblem IV]{Modellproblem IV: Overlap-Fermionen in der Gittereichtheorie}

Die Gittereichtheorie befasst sich mit der Diskretisierung der
Quantenchromodynamik (QCD). QCD ist die Theorie der starken
Wechselwirkung zwischen den Quarks.

\begin{defn}
  Die \emph{Wilson-Fermi-Matrix} $M$ beschreibt eine
  N\"achste-Nachbar-Kopplung auf einem \"aquidistanten
  $4$-dimensionalen Gitter
  \begin{equation*}
    G=\bigl\{ x\in\{1,\dots,N\}^4\bigr\}
    \qquad
    e_1=(1,0,0,0),\ e_2=(0,1,0,0),\ \dots
  \end{equation*}
  durch
  \begin{equation}
    \label{eq:wilson-fermi-matrix}
    \begin{aligned}
      (M\psi)_x
      =
      \psi_x-\kappa\biggl(
      & \sum_{\mu=1}^{4}
      \bigl(U_{x}^\mu\otimes (I-\gamma_\mu)\bigr)
      \psi_{x-e_\mu}
      \\
      +
      & \sum_{\mu=^1}^{4}
      \bigl(U_{x+e_\mu}^\mu\otimes(I+\gamma_\mu)\bigr)
      \psi_{x+e_\mu}
      \biggr)
    \end{aligned}
  \end{equation}
  mit $\psi_x\in\co^{12}$ (12 Unbekannte pro Gitterpunkt),
  $\gamma_\mu\in\co^{4\times 4}$,
  $U_x^\mu\in\mathord{\mathrm{SU}}(3)$ (unit\"are Matrizen mit Determinante +1).
   Dabei ist $\kappa\in(0,4)$ anpassbar und $U_x^\mu$
  zuf\"allig erzeugt. Gleichung \eqref{eq:wilson-fermi-matrix} ist an den
  R\"andern von $G$ periodisch zu verstehen.

  Es ist also $M\in\cnn$ mit $n=12\cdot N^4$.
  Heutzutage (Sommer 2024) liegt $N$ zwischen $32$ und $256$.
  \begin{center}
    \begin{tabular}{|r|l@{ }r|}\hline
      $N$   & \multicolumn{2}{c|}{$n$} \\ \hline
      $32$  & ca. & $1.2\cdot 10^7$ \\
      $64$  & ca. & $  2\cdot 10^8$ \\
      $128$ & ca. & $  3\cdot 10^9$ \\ \hline
    \end{tabular}
  \end{center}
\end{defn}


Beim sog.\ {\em Overlap-Fermionen-Modell} muss man
Systeme der Gestalt
\begin{equation}
  \label{eq:overlap-fermionen}
  \bigl(I+m\,\Gamma_{\!5} \sgn(Q)\bigr)\psi = \phi
\end{equation}
l�sen mit
\[
\Gamma_5 = I_{\frac{n}{4}} \otimes \gamma_5, \enspace \gamma_5 = \diag(1,1,-1-1),
\]
$Q=\Gamma_{\!5}M$ hermitesch, $Q=V\Lambda V^H$ und $m\lesssim 1$.

Dabei ist $\sgn(Q)$ definiert als $\sgn(Q)=V\sgn(\Lambda)V^H$ mit
\[
	\sgn(\Lambda)=\diag(\sgn(\lambda_i))
\]
f\"ur $\Lambda=\diag(\lambda_i)$.

\begin{bem}
  \begin{enumerate}
  \item Es gilt $\Gamma_{\!5}=\Gamma_{\!5}^H=\Gamma_{\!5}^{-1}$
    (siehe Physik) und $\sgn(Q)\cdot\sgn(Q)=I$ (falls $Q$ regul\"ar)
    mit $\sgn(Q)=(\sgn(Q))^H$. Also sind $\Gamma_{\!5}$ und
    $\sgn(Q)$ unit\"ar, also auch $\Gamma_{\!5}\cdot\sgn(Q)$.
    Auf \eqref{eq:overlap-fermionen} kann also das Verfahren vom Ende
    des letzten Abschnittes angewendet werden.
  \item $\sgn(Q)$ kann man praktisch nicht explizit berechnen. Der
    Aufwand w"are $O(n^3)$ f\"ur die Zeit und $O(n^2)$ f\"ur den
    Speicher.  F\"ur ein KUV   zur L"osung von \eqnref{eq:overlap-fermionen}
    gen\"ugt es aber, wenn man in jedem
    Schritt $\sgn(Q)y$ berechnen kann.
  \end{enumerate}
\end{bem}


$\sgn(Q)y$ kann man mit Hilfe des Lanczos-Verfahrens berechnen:

Es ist
\begin{align*}
  \sgn(Q)  &= (Q^2)^{-1/2}Q \\
  \sgn(Q)y &= (Q^2)^{-1/2}Qy .
\end{align*}
Sei $V_m=(v^1|\dots|v^m)$ die Matrix mit den Lanczos-Vektoren f\"ur $Q^2$
mit Start $v^1=Qy$ und $\norm{Qy}=1$. Dann gilt
\begin{equation*}
  Q^2V_m = V_{m+1}T_{m+1,m}.
\end{equation*}
Daraus folgt $V_m^HQ^2V_m = T_{m,m}$ (tridiagonal).

\smallskip

\textbf{Idee:} Approximiere
\begin{align*}
  \sgn(Q)y
  &= (Q^2)^{-1/2}Qy
  \\
  &\approx V_m(T_{m,m}^{-1/2})\underbrace{V_m^HQy}_{=e^1}
\end{align*}
Dabei kann man $T_{m,m}^{-1/2}$ f\"ur moderates $m$ explizit durch
Berechnung der vollst\"andigen Basis aus Eigenvektoren bestimmt werden.

\begin{sa}
  Es gilt
  \begin{equation*}
    \norm{\sgn(Q)y-V_m(T_{m,m}^{-1/2})V_m^HQy}_2
    \leq
    \norm{r^m}_2
  \end{equation*}
  mit $r^m={}$Residuum von CG f\"ur $Q^2x=y$, Startwert $x^0=0$.
\end{sa}

\begin{proof}
  Man verwendet
  \begin{enumerate}
  \item
    \begin{align*}
      a^{-1/2}
      &=
      \frac{2}{\pi}\int_0^{\infty}(a+t^2)^{-1}\;dt
      \\
      \Longrightarrow\;(Q^2)^{-1/2}
      &=
      \frac{2}{\pi}\int_0^\infty (Q^2+t^2I)^{-1}\;dt
    \end{align*}
  \item Ist $r_t^m$ das Residuum von CG f\"ur $(Q^2+t^2I)x=y$, so gilt
    $\norm{r_t^m}_2\leq\norm{r_0^m}_2$.\\
    Es gilt sogar: $r_t^m=\psi_{t,m}r_0^m$ mit $\abs{\psi_{t,m}}<1$.
  \end{enumerate}

  Der Beweis wird nur skizziert.
Wir leiten zuerst die Absch�tzung f�r die Residuen her:
$$r_i^k=p_{i,k}(Q^2+ t^2 I)b \perp K_k(Q^2). $$
Damit gibt es also ein Polynom $p_{k,0}$ mit
$$p_{k,i}(Q^2+t^2 I)b = \Phi_{k,t} p_{k,0}(Q^2)b .$$
Wenn man das Polynom nun in $t$ betrachtet ergibt dies
\begin{align*}
p_{k,t} ( s + t^2) &= \Phi_{k,t} p_k(s) \\
p_{k,t} ( s ) &= \Phi_{k,t} p_k(s-t^2)\\
\end{align*}
und da $p_{k,t}(0) = 1$, ist $\Phi_{k,t} = (p_k(-t^2))^{-1}$. Die Nullstellen der
Polynome sind die Ritz-Werte und daher $>0$.  Wegen $p(0) =1$ folgt so $p(s) > 1 $
f�r $s < 0$ und damit $|\Phi_{k,t}| < 1$. \medskip

Nun ist noch die Absch�tzung zu beweisen. Dazu betrachtet man
\begin{eqnarray*}
\lefteqn{\sgn(Q)b - Q V_m T_m^{-\frac 12} V_m^H b} \\
  &=& \frac{2}{\pi} \int_0^{\infty} Q(Q^2 + t^2I)^{-1}b - Q V_m(t^2I+T_m)^{-1}V_m^H b \; dt\\
  &=& \frac{2}{\pi} \int_0^{\infty} Q(Q^2 + t^2I)^{-1} \underbrace{\left[ b - (Q^2+t^2I) V_m(t^2I+T_m)^{-1}V_m^Hb\right]}_{= \psi_{t,m}r^m} \; dt.
\end{eqnarray*}
Weiter folgt wegen $| \psi_{t,m} \leq 1|$, dass alle Eigenwerte der hermiteschen Matrix
$$
\frac{2}{\pi} \int_0^{\infty} \psi_{t,m} Q(Q^2 + t^2I)^{-1} \, dt
$$
den Betrag $\leq 1$ haben. Damit ist auch die 2-Norm $\leq 1$.
\end{proof}



%%% Local Variables:
%%% mode: latex
%%% TeX-master: "Iterationsverfahren03"
%%% End:
