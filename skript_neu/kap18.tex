\section[Analyse allgemein]{Analyse von Zweigitterverfahren allgemein}

Gegeben seien $Ax=b$ (\glqq{}feines Gitter\grqq{}), $\hat A$ (grobes
Gitter), $P,R$ (Prolongation, Restriktion) und $S_{\omega,\nu}$
(Gl\"atter).

Iterationsoperator des Zweigitterverfahrens:
$(I-P\hat A^{-1}RA)S_{\omega,\nu}$.

\begin{defn}
  $S_{\omega,\nu}$ besitzt die \emph{Gl\"attungseigenschaft}, falls gilt
  \begin{equation*}
    \norm{AS_{\omega,\nu}}_2
    \leq
    \norm{A}_2\eta(\nu)
    \qquad
    \text{mit }
    \eta(\nu)\xrightarrow{\nu\to\infty}0.
  \end{equation*}
\end{defn}

\begin{defn}
  $P$, $R$, $\hat{A}$ besitzen die \emph{Approximationseigenschaft},
  falls gilt
  \begin{equation*}
    \norm[big]{A^{-1}-P\hat A^{-1}R}_2
    \leq
    C\frac{1}{\norm{A}_2}.
  \end{equation*}
\end{defn}

\begin{sa}
  Sind Gl\"attungs- und Approximationseigenschaft erf\"ullt, so
  konvergiert das Zweigitterverfahren, wenn $\nu$ gro\ss{} genug ist
  ($\eta(\nu)C<1$).
\end{sa}

\begin{bem}
  Die Approximationseigenschaft kann man i.A.\ nur dadurch nachweisen,
  dass man ausnutzt, dass $A$ eine endlichdimensionale Approximation
  eines kontinuierlichen Differential-Operators ist, z.B.\ bei Finiten
  Elementen (wir machen das hier nicht).
\end{bem}

F\"ur die Gl\"attungseigenschaft haben wir bereits Lemma~\ref{smooth2_lem}.
Ein weiteres Resultat ist:

\begin{lem}
  Sei $A=M-N$. Es gelte in einer Norm $\norm{}$
  \begin{equation*}
    \norm{I-2M^{-1}A} \leq 1
    \qquad\text{und}\qquad
    \norm{M} \leq C\norm{A}
    .
  \end{equation*}
  Dann gilt mit dem Gl\"atter $S=M^{-1}N=I-M^{-1}A$
  \begin{equation*}
    \norm[big]{AS^\nu}
    \leq
    \sqrt{\frac{2}{\pi\nu}}\;\norm[big]{A}.
  \end{equation*}
\end{lem}

\begin{proof}
  Wir zeigen zun\"achst: F\"ur $B$ mit $\norm{B}<1$ gilt
  \begin{equation*}
    \norm[big]{ (I-B) (I+B)^\nu }
    \leq
    2 \binom{\nu}{\lfloor\nu/2\rfloor}
    \leq
    2^{\nu-1}\sqrt{\frac{2}{\pi\nu}}.
  \end{equation*}
  Die zweite Ungleichung folgt mit der Stirlingschen Formel.
  Wir verwenden
  \begin{align*}
    (I-B) (I+B)^\nu
    &=
    (I-B)\sum_{\mu=0}^{\nu}\binom{\nu}{\mu}B^\mu
    \\
    &=
    I+\sum_{\mu=1}^{\nu}
    \Biggl(\binom{\nu}{\mu}-\binom{\nu}{\mu-1}\Biggr)
    B^\mu-B^{\nu+1}.
  \end{align*}
  Mit
  $\dbinom{\nu}{\mu}=\dbinom{\nu}{\nu-\mu}$
  und
  $\dbinom{\nu}{\mu}\geq\dbinom{\nu}{\mu-1}$
  f\"ur $\mu=1,\dots,\lfloor\nu/2\rfloor$ ergibt sich
  \begin{align*}
    \norm[big]{ (I-B)(I+B)^\nu }
    &\leq
    \underbrace{\norm[big]{ I-B^{\nu+1} }}_{\leq2}
    +
    2\sum_{\mu=1}^{\lfloor\nu/2\rfloor}
    \Biggl(\binom{\nu}{\mu}-\binom{\nu}{\mu-1}\Biggr)
    \\
    &\leq
    2+2\Biggl(\binom{\nu}{\lfloor\nu/2\rfloor}-\binom{\nu}{0}\Biggr)
    \\
    &=
    2\binom{\nu}{\lfloor\nu/2\rfloor}.
  \end{align*}
  Zum Beweis des Lemmas: Es ist
  \begin{equation*}
    M^{-1}AS^\nu
    =
    \frac{1}{2^{\nu+1}}(I-B)(I+B)^\nu
    \qquad
    \text{mit }
    B=I-2M^{-1}A
  \end{equation*}
  und damit
  \begin{equation*}
    \norm{AS^\nu}
    \leq
    \norm{M}\cdot\norm{M^{-1}AS^\nu}
    \leq
    \norm{M}\frac{1}{2^{\nu+1}}2^{\nu+1}\sqrt{\frac{2}{\pi\nu}}.
  \end{equation*}
\end{proof}

%%% Local Variables: 
%%% mode: latex
%%% TeX-master: "Iterationsverfahren03"
%%% End:
