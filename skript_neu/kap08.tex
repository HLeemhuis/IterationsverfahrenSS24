\section{Konvergenzanalyse f�r GMRES/MINRES}

Gegeben sei $A\in\cnn,\ Ax=b,\ A$ regul�r.

\begin{defn}
Der \emph{numerische Wertebereich} von $A$ ("`fields of value"') ist definiert durch
\begin{align*}
F(A)&=\left\{\frac{\langle Ax,x\rangle}{\langle x,x\rangle}:\ x\in \cn,\ x\ne 0 \right\}\\
&=\left\{\langle Ax,x\rangle:\ \|x\|_2=1 \right\}.
\end{align*}
\end{defn}

Mit elementarer Rechnung ergeben sich die folgenden Eigeschaften des numerischen Wertebereichs:
\begin{lem}
\begin{enumerate}
\item Ist $A$ hermitesch mit spek$(A)=\{\lambda_1\le\ldots\le\lambda_n\}$, dann ist $F(A)=[\lambda_1,\lambda_n]$.
\item spek$(A)\subseteq F(A)$.
\item Ist $A$ normal, dann ist $F(A)=$co(spek$(A)$) (co$(M)$ = konvexe H�lle von $M$).
\end{enumerate}
\end{lem}

\begin{sa}
\begin{enumerate}
\item Der numerische Wertebereich $F(A)$ ist kompakt, also ist der \emph{numerische Radius}
\[
r(A):=\sup \{|\lambda|:\ \lambda\in F(A)\}
\]
endlich (und das $\sup$ ist ein $\max$).
\item $F(A)$ ist konvex.
\end{enumerate}
\end{sa}
\begin{proof}
\begin{enumerate}
\item Klar.
\item Das ist etwas weniger selbstverst\"andlich. Sei $\lambda, \mu \in F(A)$, $\lambda \neq \mu$. Wir zeigen, dass $\eta(t) =t\lambda +(1-t) \mu \in F(A)$ f�r alle $t \in [0,1]$. Man beachte, dass gilt
\[
\eta(t) \in F(A) \Leftrightarrow t \in F(\underbrace{\alpha I+\beta A}_{ =: S}) \text{ mit } \alpha = -\frac{\mu}{\lambda-\mu}, \beta = \frac{1}{\lambda-\mu}.
\]
Wir zeigen deshalb, dass $[0,1] \in F(S)$.
Sei nun $\lambda = \langle Ax, x \rangle, \mu = \langle Ay, y \rangle$ with $\|x\|= \|y\|=1$.
Eine einfache Rechnung ergibt
\[
\langle Sx, x \rangle =1, \langle Sy, y \rangle=0.
\]
Definiere $g:\mathbb{R} \to \mathbb{C}$ mit $\theta \mapsto g(\theta)= \langle Sy,x\rangle e^{-i\theta}+ \langle Sx,y \rangle e^{i\theta}$. Offensichtlich ist $\Im(g(0))= -\Im(g(\pi))$. Weil $g$ und damit $\Im(g)$ stetig ist, existiert $\theta_0 \in [0,\pi]$ mit $\Im(g(\theta_0))=0$.

Wegen $\lambda \neq \mu$ sind die Vektoren $y$ und $\hat x = e^{i\theta_0}x$ linear unabh\"angig. Also gilt $z(t) = t \hat x + (1-t)y \neq 0$ f\"ur alle $t$ und wir bezeichnen mit $\hat z(t)$ ihre auf 1 normierte Versionen. Die Funktion $f(t) = \langle S \hat z(t), \hat z(t)$ is stetig und reellwertig (!) auf $[0,1]$ mit $f(0) = 0, f(1) =1$. Also gilt $[0,1] \in F(S)$. Dies war zu zeigen.
\end{enumerate}
\end{proof}
\begin{bsp}Die Matrix $A$ aus Beispiel~\ref{stagnation_bsp} ist normal. Deshalb ist $F(A)$ die konvexe H"ulle der Eigenwerte, also hier ein (ausgef"ulltes) regelm"a"siges $n$-Eck im Einheitskreis.  \end{bsp}

\begin{aufg}Zeigen Sie, dass f"ur die Matrix \[A=\left(\begin{array}{cc}0&2\\0&0\end{array}\right)\] der numerische Wertebereich gegeben ist durch $F(A)=\{z\in\co:\ |z|\le 1\}$.\end{aufg}
\begin{lem}Sei $V_m\in\co^{n\times m},\ V_m^HV_m=I$. Dann gilt
\[
F(V_m^HAV_m)\subseteq F(A).
\]
\end{lem}
Der Beweis ist wieder trivial.
 
\begin{sa}\label{GMRES_NST_sa}
Sei $p_m^{\text{GMRES}}$ das Residuen-Polynom aus dem GMRES-Verfahren, d.h. 
\[
r^m=p_m^{\text{GMRES}}(A)r^0, \quad p_m^{\text{GMRES}}\in\overline{\Pi} _m.
\]
Dann gilt 
\[
p_m^{\text{GMRES}}(\xi)=0\Rightarrow\frac{1}{\xi}\in F(W_m^HA^{-1}W_m)\subseteq F(A^{-1}).
\]
Dabei ist $W_m\in\co^{n\times m}$ eine Matrix, deren Spalten eine Orthonormalbasis von $A\cdot K_m(A,r^0)$ bilden.
\end{sa}
\begin{proof}
Setze $p_m:=p_m^{\text{GMRES}}$. Es ist
\begin{align*}
p_m(t)&=\prod\limits_{j=1}^m\left(1-\frac{t}{\xi_j} \right).
\intertext{Betrachte $\xi=\xi_m$:}
p_m(t)&=\left(1-\frac{t}{\xi} \right)\cdot q_{m-1}(t)\text{ mit } q_{m-1}(t)=\prod\limits_{j=1}^{m-1}\left(1-\frac{t}{\xi_j} \right).
\intertext{Damit ist}
\|p_m(A)r^0\|_2&=\|(I-\frac{1}{\xi}A)\cdot \underbrace{q_{m-1}(A)\cdot r^0}_{w}\|_2
\end{align*}
minimal. Also ist $\xi$ so, dass $\|(I-\frac{1}{\xi}A)w\|_2$ minimal ist.
Nach Lemma~\ref{minimierungs_lem} (s. unten) gilt deshalb
\[
\frac{1}{\xi}=\frac{\langle w,Aw\rangle}{\langle Aw,Aw\rangle}=\frac{\langle A^{-1}y ,y\rangle}{\langle y,y\rangle}
\]
mit $y=Aw\in A\cdot K_m(A,r^0)$.
\end{proof}

\begin{lem} \label{minimierungs_lem}
Die L"osung von $\underset{\gamma}{\min}\|w-\gamma y\|_2$ ist
\[
\gamma^*=\frac{\langle w,y\rangle}{\langle y,y\rangle}
\]
\end{lem}
%{\bf Beweis:}
\begin{proof}
Es ist
\begin{align*}
\langle w-\gamma y,w-\gamma y\rangle&=\langle w,w\rangle-\gamma \langle y,w\rangle
						-\bar \gamma \langle w,y\rangle+\gamma\bar \gamma\langle y,y\rangle\\
&=\langle w,w\rangle+\langle y,y\rangle\left(\gamma\bar\gamma-\frac{\langle y,w\rangle}{\langle y,y\rangle}\gamma 
						-\frac{\langle w,y\rangle}{\langle y,y\rangle}\bar \gamma\right.\\
&\hspace*{4cm}\left.				+\frac{\langle y,w\rangle\langle w,y\rangle}{\langle y,y\rangle^2}\right)
						-\frac{\langle y,w\rangle\langle w,y\rangle}{\langle y,y\rangle}\\
&=\langle w,w\rangle+\langle y,y\rangle\left(\gamma-\frac{\langle w,y\rangle}{\langle y,y\rangle} \right)
					    \left(\bar \gamma-\frac{\langle y,w\rangle}{\langle y,y\rangle} \right)\\
&\hspace*{7cm}	-\frac{\langle y,w\rangle\langle w,y\rangle}{\langle y,y\rangle}.
\end{align*}
Das Minimum wird also f�r
\[
\gamma ^*=\frac{\langle w,y\rangle}{\langle y,y\rangle}
\]
erreicht.
%}
\end{proof}

Schr�nken wir $A\in\co^{n\times n}$ auf einen Unterraum $V \subseteq \co^n$ ein, so ist nicht
notwendig $AV \subseteq V$. M�chte man einen Homomorphismus auf $V$, muss man $AV$ geeignet
projizieren.

\begin{defn}Die {\em orthogonale Projektion} von $A \in \co^{n \times n}$ auf den Unterraum
 $V \subseteq \co^n$ entsteht durch
orthogonale Projektion von $AV$ auf $V$. Bilden die Spalten von
$ V_m\in\co^{m\times n}$ eine Orthonormalbasis von $V$  ($V_m^HV_m=I$), so ist
$V_m^HAV_m$ die Darstellung der orthogonalen Projektion von $A$ auf $V$
in der Basis "`Spalten von $V_m$"'.
\end{defn}

\begin{sa}\label{GMRESabschaetz_sa}
Es sei $0\notin F(A)$ und $c:=\min\{|\lambda|:\ \lambda\in F(A)\}>0$. Dann ist $c \leq \|A\|_2$
und es gilt
\begin{equation} \label{gmres_konv_sa}
\|p_m^{\text{GMRES}}(A)r^0\|_2\le \left(1-\textstyle\frac{c^2}{\|A\|_2^2} \right)^{\textstyle\frac{m}{2}}\cdot \|r^0\|_2.
\end{equation}
\end{sa}
\begin{proof} Es ist
\[
c \le \frac{| \langle Aw,w\rangle|}{\langle w,w \rangle} \leq
 \frac{ \|Aw\|_2 \cdot \|w\|_2}{\langle w,w \rangle} \leq
\frac{ \|A\|_2  \cdot \|w\|_2^2}{\langle w,w \rangle} = \|A\|_2.
\]

Zum Beweis von \eqnref{gmres_konv_sa} starten wir mit einer
Vor�berlegung.
Nach Lemma~\ref{minimierungs_lem} gilt f�r $\frac{1}{\xi} =
\frac{\langle w, Aw\rangle}{\langle Aw,Aw \rangle}$
\[
\langle w-\textstyle\frac{1}{\xi}Aw,w-\textstyle\frac{1}{\xi}Aw\rangle
=\langle w,w\rangle-\textstyle\frac{\langle Aw,w\rangle\langle w,Aw\rangle}{\langle Aw,Aw\rangle} .
\]
F�r $w\ne 0$ gilt au�erdem
\begin{align*}
\langle Aw,Aw\rangle&=\|Aw\|_2^2\le \|A\|_2^2\cdot \langle w,w\rangle,\\
|\langle Aw,w\rangle|^2&\ge \left|\textstyle\frac{\langle Aw,w\rangle}{\langle w,w\rangle} \right|^2\cdot \langle w,w\rangle^2\\
&\ge c^2\cdot \langle w,w\rangle^2.
\end{align*}
Also ist f�r die angegebene Wahl von $\xi$
\begin{equation} \label{absch1_eq}
\langle w-\textstyle\frac{1}{\xi}Aw,w-\textstyle\frac{1}{\xi}Aw\rangle\le  \left(1-\textstyle\frac{c^2}{\|A\|_2^2} \right)\cdot \langle w,w\rangle,
\end{equation}
mit $ \left(1-\textstyle\frac{c^2}{\|A\|_2^2} \right)\in[0,1)$.

\medskip
Wir schreiben wieder $p_m$ statt $p_m^{\text{GMRES}}$
und zeigen \eqnref{gmres_konv_sa} per Induktion.
\begin{Blist}{$m-1\to m$}
\item[\hfill $m=0$] klar.
\item[$m-1\to m$] F�r $w=p_{m-1}(A)r^0$ ist nach Induktionsvoraussetzung
\[
\|w\|_2\le \left(1-\textstyle\frac{c^2}{\|A\|_2^2} \right)^{\textstyle\frac{m-1}{2}}\cdot \|r^0\|_2.
\]
Auf Grund der Minimalit�tseigenschaft ist f�r alle $\xi\in\co\backslash\{0\}$
\[
\|p_m(A)r^0\|_2\le \|(I-\textstyle\frac{1}{\xi}A) p_{m-1}(A)r^0\|_2 = \|(I-\textstyle\frac{1}{\xi}A)w\|_2.
\]
Speziell f�r die Wahl $\textstyle\frac{1}{\xi}=\textstyle\frac{\langle w,Aw\rangle}{\langle Aw,Aw\rangle}$ ergibt sich nach \eqnref{absch1_eq}
\[
\left \|\left(I-\textstyle\frac{1}{\xi}A\right)w \right\|_2^2\le \left(1-\textstyle\frac{c^2}{\|A\|^2} \right) \cdot \|w\|_2^2.
\]
\end{Blist}
\end{proof}

\begin{bem}Der Beweis hat gezeigt, dass unter den Voraussetzungen von Satz \ref{GMRESabschaetz_sa} sogar gilt, dass GMRES($k$) f�r alle Restartwerte
$k$  mit der selben Absch�tzung \eqnref{gmres_konv_sa} konvergiert.
\end{bem}

Wir diskutieren die Nullstellen der GMRES-Polynome jetzt noch genauer.

\begin{lem}\label{GMRES_res_lem}
F�r die GMRES-Residuen
\[
r^m=p_m^{\text{GMRES}}(A)r^0
\]
gilt
\[
r^m\bot A\cdot K_m(A,r^0).
\]
(Zum Vergleich: Beim cg-Verfahren gilt $r^m \bot K_m(A,r^0)$.)
\end{lem}
\begin{proof}
Es gilt
\[
r^m=b-Ax^m=r^0-AV_mz^m, \quad z^m\in\co^m\text{ mit } \|r^0-AV_mz^m\|_2\text{ minimal.}
\]
Also erf�llt $z^m$ die Normalengleichung (s.\ Numerik I)
\[
\underbrace{V_m^HA^HAV_m}_{M_m}z^m=V_m^HA_m^Hr^0
\]
mit $M_m$ hpd, da $V_m$ vollen Rang besitzt. Also ist
\[
z^m=M_m^{-1}V_m^HAr^0
\]
und es gilt f�r alle $y\in\co^{m}$
\begin{align*}
\langle r^m,AV_my\rangle
&=\langle r^0-AV_mM_m^{-1}V_m^HA^Hr^0, AV_my\rangle\\
&=\langle r^0,AV_my\rangle-\langle r^0,AV_m\underbrace{M_m^{-1}V_m^HA^HAV_m}_{=I} y\rangle\\
&=0.
\end{align*}
\end{proof}

\begin{sa}
Die Nullstellen $\xi_i$ der GMRES-Polynome $p_m^{\text{GMRES}}$ sind Eigenwerte des verallgemeinerten
Eigenwertproblems
\begin{equation}
(H_{m+1,m}^HH_{m+1,m})\cdot z=\xi H_m^H \cdot z,\label{verallg_ewprob_eq}
\end{equation}
wobei $H_m$ aus $H_{m+1,m}$ durch Streichen der letzten Zeile entsteht.
\end{sa}
\begin{proof}
Man beachte, dass das Produkt $H_{m+1,m}^HH_{m+1,m}$  hpd ist, $H_m$ jedoch singul�r sein kann.
Sei 
\[
r^m=p_m(A)r^0=\left(I-\textstyle\frac{1}{\xi}A \right)\underbrace{q_{m-1}(A)r^0}_{= V_mz},\ p_m(\xi)=0.
\]
Nach Lemma \ref{GMRES_res_lem} gilt mit $AV_m = V_{m+1}H_{m+1,m}$
\begin{align*}
&r^m \bot AK_m(A,r^0)\\
\iff&\langle r^m,AV_my\rangle=0 \quad \forall\ y,\\
\iff&\langle (I-\xi^{-1}A)V_mz,AV_my\rangle =0 \quad \forall\ y, \\
\iff&\langle V_mz-\xi^{-1}V_{m+1}H_{m+1,m}z,V_{m+1}H_{m+1,m}y\rangle=0 \quad \forall\ y, \\
\iff&\langle V_mz,V_{m+1}H_{m+1,m}y\rangle-\xi^{-1}\langle V_{m+1}H_{m+1,m}z,V_{m+1}H_{m+1,m}y\rangle=0 \quad \forall\ y, \\
\iff&y^HH_{m+1,m}^HV_{m+1}^HV_mz-\xi^{-1}y^HH_{m+1,m}^H\underbrace{V_{m+1}^HV_{m+1}}_{=I}H_{m+1,m}z=0 \quad \forall\ y. \\
\intertext{Dabei ist $V_{m+1}^HV_m=\left(\begin{array}{c}
I_m\\
\begin{array}{ccc}
0&\ldots&0
\end{array}
\end{array}\right)$ und damit $H_{m+1,m}^HV_{m+1}^HV_m=H_m^H$. Wir erhalten also}
&r^m \bot AK_m(A,r^0)\\
\iff&y^H(H_m^Hz-\xi^{-1}H_{m+1,m}^HH_{m+1,m}z)=0\\
\iff&H_m^Hz-\xi^{-1}H_{m+1,m}^HH_{m+1,m}z=0.
\end{align*}
\end{proof}

\textbf{Bedeutung von \eqnref{verallg_ewprob_eq}:} Betrachte die orthogonale Projektion von $A^{-1}$ auf
$A\cdot K_m(A,r^0)$.
Eine ONB von $AK_m(A,r^0)$ sind die Spalten von
\[
AV_mM_m^{-\textstyle\frac{1}{2}} \quad \text{mit } M_m=V_m^HA^HAV_m,
\]
denn (Basis ist klar)
\[
M_m^{-\textstyle\frac{1}{2}}\underbrace{V_m^HA^HAV_m}_{M_m}M_m^{-\textstyle\frac{1}{2}}=I.
\]
Die orthogonale Projektion von $A^{-1}$ auf $AK_m(A,r^0)$ wird repr�sentiert durch
\begin{align*}
M_m^{-\textstyle\frac{1}{2}}V_m^H(A^HA^{-1}A)V_mM_m^{-\textstyle\frac{1}{2}}
&=M_m^{-\textstyle\frac{1}{2}}(AV_m)^HV_mM_m^{-\textstyle\frac{1}{2}}\\
&=M_m^{-\textstyle\frac{1}{2}}(V_{m+1}H_{m+1,m})^HV_mM_m^{-\textstyle\frac{1}{2}}\\
&=M_m^{-\textstyle\frac{1}{2}}H_m^HM_m^{-\textstyle\frac{1}{2}}.
\end{align*}
Damit ist $\xi^{-1}$ Eigenwert dieser orthogonalen Projektion, wenn
\begin{alignat*}{6}
&&M_m^{-\textstyle\frac{1}{2}}H_m^HM_m^{-\textstyle\frac{1}{2}}z&=\xi^{-1} z &&\quad \text{mit } z\ne 0\\
\iff&&H_m^Hy&=\xi^{-1}M_my &&\quad\text{mit } y=M_m^{-\textstyle\frac{1}{2}}z.
\end{alignat*}

\bigskip

Wir halten fest
\begin{sa}
$\xi$ ist Nullstelle von $p_m^{\text{GMRES}}$ genau dann, wenn $\xi^{-1}$ Eigenwert der orthogonalen Projektion von
$A^{-1}$ auf $A\cdot K_m(A,r^0)$ ist.
\end{sa}

Spezialfall: $A$ normal
\begin{eqnarray*} \begin{array}{lcl}
  \Rightarrow  A &=& Q^H \Lambda Q, \quad Q^H Q=I, \quad \Lambda =\diag(\lambda_1,\ldots,\lambda_n) \\
  \Rightarrow  A^{-1} &=& Q^H \Lambda^{-1} Q
\end{array} \end{eqnarray*}
Damit ist die orthogonale Projektion von $A^{-1}$ auf $AK_m(A,r^0)$ die Matrix
\begin{align*}
W_m^H A^{-1} W_m = W_m^H Q^H \Lambda^{-1}\underbrace{QW_m}_{\text{orthonormal}} =:B \in \cmm
\end{align*}
mit $W_m^HW_m=I,W_m\in\cnm$.
$B$ ist selbst normal, $B=P^HD_mP, P\in\cmm$ orthonormal, $D_m=\diag(d_1,\ldots,d_m)$. \\ Also
\begin{eqnarray*}
 & P^HD_mP &= W_m^HQ_m^H\Lambda^{-1}QW_m \\
\Leftrightarrow & D_m &= (PW_m^HQ_m^H)\Lambda^{-1}(\underbrace{QW_mP^H}_{=:Y}) \\
\Rightarrow & d_i &= \sum\limits_{j=1}^n \frac{1}{\lambda_i} |y_{ji}|^2  \quad \text{mit}
\sum\limits_{j=1}^n |y_{ji}|^2 =1.\end{eqnarray*}
$\frac{1}{\xi}$ ist einer der Eigenwerte $d_i$ von $B$. Also haben wir
\begin{sa}
Ist $A$ normal (z.B. $A=A^H$), so sind die Nullstellen der GMRES-Polynome gewichtete harmonische Mittel
der Eigenwerte von $A$, d.h.
$$ p_m^{\text{GMRES}}(\xi)=0 \Rightarrow \frac{1}{\xi}=\sum\limits_{j=1}^n\alpha_j \frac{1}{\lambda_j} $$
mit $\alpha_j>0,\sum\alpha_j=1,\lambda_j$ Eigenwerte von A.
\end{sa}
Man nennt die $\xi$ deshalb auch {\em harmonische Ritz-Werte}.
\begin{bem}
$A$ sei hermitesch, $spek(A)\cap[-b,a]=\emptyset,a,b>0$. Dann folgt $\xi\notin[-b,a]$.
\end{bem}
Spezielle Resultate f"ur MINRES:
\begin{sa}
Sei $A=A^H \in\cnn$ regul"ar, $\lambda_1 \leq \lambda_2 \leq \ldots \leq \lambda_{n^*}$ seien die
verschiedenen Eigenwerte von $A$, $n^*\leq n$.
\begin{enumerate}
\item Es gelte $\lambda_1 \leq \ldots \leq \lambda_k < 0$ und $\lambda_{k+1}>0$.
      Dann gilt f"ur das Verfahren MINRES
	\begin{eqnarray*}
	\|r^{m+k}\|_2 \leq \frac{2c^m}{1+c^{2m}}\prod_{i=1}^k           \left(1-\frac{\lambda_{n^*}}{\lambda_i}\right)\|r^0\|_2
      \end{eqnarray*}
	mit $c=\frac{\sqrt{\kappa}-1}{\sqrt{\kappa}+1} \text{ und } \kappa=\frac{\lambda_{n^*}}{\lambda_{k+1}}$
\item Es gelte $\spek(A)\subseteq[-b,-a]\cup[a,b],0<a\leq b$. Dann gilt f"ur das Verfahren MINRES
	\begin{eqnarray*}
	\|r^{2m}\|_2 \leq \frac{2\tilde c^m}{1+\tilde c^{2m}}\|r^0\|_2
	\end{eqnarray*}
	mit $\tilde c=\frac{\tilde \kappa-1}{\tilde \kappa+1} \text{ und } \tilde \kappa=\frac{b}{a}$
\end{enumerate}
\end{sa}
\begin{proof}
Zu 1.: Nehme
\begin{eqnarray*}
p_{m+k}(t)=\prod_{i=1}^k \left(1-\frac{t}{\lambda_i}\right) \cdot T_m^{[\lambda_{k+1},\lambda_{n^*}]}(t)
\end{eqnarray*}
mit $T_m^{[\lambda_{k+1},\lambda_{n^*}]}$ Tschebyscheff-Polynom f"ur $[\lambda_{k+1},\lambda_{n^*}]$.
Dann gilt mit Korollar~\ref{Tscheb_pos_def_cor}
\begin{eqnarray*}
\max_{t\in \spek(A)}|p_{m+k}(t)| &=& \max\limits_{t\in[\lambda_{k+1},\lambda_{n^*}]}|p_{m+k}(t)| \\
 &\leq& \frac{2c^m}{1+c^{2m}}\prod\limits_{i=1}^k\left(1-\frac{\lambda_{n^*}}{\lambda_i}\right). \\
\end{eqnarray*}
Damit gilt
\begin{eqnarray*}
\|r^{m+k}\|_2 &=& \|p_{m+k}^{\text{MINRES}}(A)r^0\|_2 \\              & \leq & \|p_{m+k}(A)r^0\|_2 \\
 &\leq& \|p_{m+k}(A)\|_2\cdot\|r^0\|_2 \\
 &=& \max_{t\in \spek(A)}|p_{m+k}(t)|\cdot\|r^0\|_2 \\
 &\leq&  \frac{2c^m}{1+c^{2m}}\cdot\prod_{i=1}^k\left(1-\frac{\lambda_{n^*}}{\lambda_i}\right)\cdot\|r^0\|_2.
\end{eqnarray*}

Zu 2.: Nehme $p_{2m}(t)=T_m^{[a^2,b^2]}(t^2)$. Dann gilt
\begin{eqnarray*}
\|r^{2m}\|_2 &\leq& \|T_m^{[a^2,b^2]}(A^2)\|_2\cdot\|r^0\|_2 \\
 &\leq& \max_{s\in \spek(A^2)}|T_m^{[a^2,b^2]}(s)|\cdot\|r^0\|_2 \\
 &\leq& \max_{s\in [a^2,b^2]}|T_m^{[a^2,b^2]}(s)|\cdot\|r^0\|_2 \\
 &=& \frac{2 \tilde c^m}{1+\tilde c^{2m}}\cdot\|r^0\|_2
\end{eqnarray*}
\end{proof}
\begin{bem}
Im 1. Fall ist f"ur kleines $k$ die Konvergenz von MINRES also vergleichbar mit CG
(auf $[\lambda_{k+1},\lambda_{n^*}]$). CG f"ur $A^HAx=A^Hb$ mit $A^HA$ hpd ben"otigt:
2 MVM pro Schritt (A und $A^H$) und $$\kappa(A^HA)=
\frac{\max\{\lambda_1^2,\lambda_{n^*}^2\}}{\min\{\lambda_k^2,\lambda_{k+1}^2\}}\geq \kappa^2.$$
Wenn $k$ klein ist, ist CG f"ur $A^HA$ also 4-mal so aufwendig. \medskip

Vergleich mit CG im 2. Fall, wobei $a$ und $b$ jetzt beide Eigenwerte von $A$ seien. \\
Wegen $\kappa(A^HA)=\frac{b^2}{a^2}$ folgt dann: CG f"ur $A^HA$ ist gleich schnell wie MINRES.
\end{bem}
Der Fall $\spek(A)\subseteq[-b,-a]\cup[c,d]$ kann allgemein mit Orthogonalpolynomen bzgl.
zueinander disjunkter Intervalle behandelt werden.\\
Siehe dazu: Bernd Fischer - Polynomial Based Iteration Methods for Symmetric
Linear Systems, Wiley (1995).