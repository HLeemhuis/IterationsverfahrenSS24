\section[Analyse von Tschebyscheff- und...]{Analyse von Tschebyscheff- und Ri\-chard\-son-Ver\-fahren}

\begin{sa} \label{Tscheb_analyse_sa}
  Sei $p_m$ das Polynom des Tschebyscheff-Verfahrens
  bzgl.\ $E(f_1,f_2,\rho)\not\ni 0$ mit $\spek(A)\subseteq E(f_1,f_2,\rho)$.
  Dann gilt
  \begin{equation*}
    r^m=p_m(A)r^0
  \end{equation*}
  und damit f�r jede Norm $\norm{\cdot}$
  \begin{equation*}
    \norm{r^m}\leq\norm{p_m(A)}\cdot\norm{r^0}.
  \end{equation*}
  \begin{enumerate}
  \item
    \label{TschVerf-analyse-satz1}
    Es sei $A$ diagonalisierbar mit $A=SDS^{-1}$ (siehe \ref{Konvergenzdiag_kor})
    und $\norm{}$ sei die Norm $\norm{}_S$.
    Dann gilt
    \begin{equation*}
      \norm{p_m(A)}
      \leq
      \frac{\rho^m+\rho^{-m}}{\,\abs{w_\gamma^m+w_\gamma^{-m}}\,},
    \end{equation*}
    wobei $w_\gamma$ (betragsgr��te)
    L�sung von $\frac{1}{2}(w+w^{-1})=-b/a$.
  \item Ist in \ref{TschVerf-analyse-satz1}. sogar $S$ orthogonal
    (z.B. weil $A$ normal ist), so gilt sogar
    \begin{equation*}
      \norm{p_m(A)}_2
      \leq
      \frac{\rho^m+\rho^{-m}}{\,\abs{w_\gamma^m+w_\gamma^{-m}}\,}.
    \end{equation*}
  \end{enumerate}
\end{sa}

\begin{proof}
  Ist $q_m$ die approximative Optimall�sung f"ur die Ellipse $E(-1,1,\rho)$ mit
  $q_m(\gamma) = 1$, dann gilt nach Satz~\ref{cmschranke_sa}
  \begin{equation*}
    \max_{t\in E(-1,1,\bar{\rho})}\abs{q_m(t)}
    =
    \frac{\bar{\rho}^m+\bar{\rho}^{-m}}{\,\abs{w_\gamma^m+w_\gamma^{-m}}\,}.
  \end{equation*}
  Mit der affinen Transformation $z\mapsto az+b$ aus Lemma~\ref{ellipsentransf_lem},
  welche $E(-1,1,\rho)$ auf
  $E(f_1,f_2,\rho)$ abbildet, ergibt sich der Satz.
\end{proof}
\medskip

Besonders wichtig ist der Fall $\spek(A)\subseteq[\xi,\Xi]\subseteq(0,+\infty)$
und $A$ symmetrisch und positiv definit:

\begin{cor} \label{Tscheb_pos_def_cor}
  Ist $\spek(A)\subseteq[\xi,\Xi]\subseteq(0,+\infty)$
  und $A$ spd, dann gilt f�r die Residuen des Tschebyscheff-Verfahrens
  \begin{equation*}
    \norm{r^m}_2\leq\norm{p_m(A)}_2\cdot\norm{r^0}_2
  \end{equation*}
  mit
  \begin{equation*}
    \norm{p_m(A)}_2
    \leq
     \frac{2c^m}{1+c^{2m}} \enspace
     \mbox{ mit } c = \frac{\sqrt{\kappa}-1}{\sqrt{\kappa}+1}, \,
    \kappa = \frac{\Xi}{\xi}.
  \end{equation*}
\end{cor}

\begin{proof}
  Hier ist $f_1=\xi$, $f_2=\Xi$, $\rho = 1$, $a= \frac{1}{2}(\Xi - \xi), b = \frac{1}{2}(\Xi+\xi)$ und $\gamma = -b/a$. Aus
Satz~\ref{Tscheb_analyse_sa} folgt so
\[
   \norm{p_m(A)}_2 \leq \frac{2}{\,\abs{w_\gamma^m+w_\gamma^{-m}}\,} 
\]
mit
\begin{equation*}
\frac{1}{2}(w_\gamma+w_\gamma^{-1})=-\frac{b}{a}=- \frac{\Xi+\xi}{\Xi-\xi}=
    -\frac{\kappa+1}{\kappa-1}.
\end{equation*}
Auf\/l�sen der quadratischen Gleichung ergibt
\begin{eqnarray*}
  w_\gamma &=& - \frac{\kappa+1}{\kappa-1} -          \sqrt{\left(\frac{\kappa+1}{\kappa-1}\right)^2-1} \\
   &=& - \frac{\kappa+1}{\kappa-1} - \sqrt{\frac{4\kappa}{(\kappa-1)^2}} \\
   &=& - \frac{\kappa+1}{\kappa-1} - \frac{2\sqrt{\kappa}}{\kappa-1} \\
   &=& - \frac{\left(\sqrt{\kappa}+1\right)^2}{(\sqrt{\kappa}-1)(\sqrt{\kappa}+1)} \\
   &=& -\frac{\sqrt{\kappa}+1}{\sqrt{\kappa}-1}.
\end{eqnarray*}

\end{proof}


Zum Abschluss (und Vergleich) analysieren wir noch das Richardson-Ver\-fahr\-en
2.~Ordnung. Dazu sei $A$ spd mit $\spek(A)\subseteq[\xi,\Xi]$.

\begin{aufg}
  F�r das relaxierte Richardson-Verfahren 1.~Ordnung
  \begin{equation*}
    x^{m+1} = x^m+\alpha r^m
  \end{equation*}
  wird f�r die zugeh�rigen Polynome $p_m(z)=(1-\alpha z)^m$ die Gr��e
  \begin{equation*}
    \max_{z\in[\xi,\Xi]}\abs{p_m(z)}
  \end{equation*}
  minimal, wenn $\alpha=2/(\xi+\Xi)$.
\end{aufg}

\begin{bem}
  Das relaxierte Richardson Verfahren 1.~Ordnung $Ax=b$ ist �quivalent zum
  gew�hnlichen Richardson-Verfahren 1.~Ordnung f"ur
  \begin{equation*}
    (\alpha A)x = \alpha b,
  \end{equation*}
  also
  \begin{equation*}
    x^{m+1} = x^m+(\alpha b-\alpha A x^m).
  \end{equation*}
  Wir gehen ab jetzt davon aus, dass diese "`Skalierung"' des
  LGS (mit $\alpha=\frac{2}{\xi+\Xi})$ bereits erfolgte, d.h.\ es gilt ab jetzt
  \begin{equation*}
    \spek(A) \subseteq [1-\gamma, 1+\gamma]
    \text{ mit }
    \gamma = \frac{\Xi-\xi}{\Xi+\xi} \in [0,1).
  \end{equation*}
\end{bem}

F�r das System
\begin{equation*}
  Ax=b
\end{equation*}
betrachten wir jetzt das Richardson-Verfahren 2.~Ordnung

%%$$\norm{p_m(A)}_{s/2}\leq \max_{\lambda\in
%%E(f_1,f_2,\rho)}\abs{p_m(\lambda)}
%%=\frac{\rho^m+\pfrac{1}{s}^m}{\abs{w_\gamma^m+\pfrac{1}{w_\gamma}^m}}$$
%%
%%\begin{equation*}
%%  \frac{1}{2}(w_\gamma+w_\gamma^{-1})=-\frac{b}{a}
%%  \text{ mit }
%%  \abs{w_\gamma} >\rho.
%%\end{equation*}
%%Diskussion der oberen Schranke:
%%
%%-----------------------------
%%
%%BILD 1
%%
%%Experimente mit falschem Intervall liefern: Es kann immer noch
%%konvergieren. Die Konvergenz ist aber am schnellsten, wenn das
%%richtige Intervall gew�hlt wird.
%%
%%
%%
%%\textbf{Richardson 1.~Ordnung}
%%
%%$$Ax=b$$
%%
%%$$(\alpha A)x=\alpha b$$
%%
%%$$\spek(A)\subseteq[1-\gamma,1+\gamma]$$
%%$$x=(I-A)x+b$$
%%%%$$x=Bx+b,\qquad \spek(B)\subseteq[-\gamma,+\gamma]$$

\begin{align*}
  x^{m+1}
  &=
  \omega\bigl((I-A)x^m+b\bigr)+(1-\omega)x^{m-1}\\
  &=
  \omega(x^m+r^m)+(1-\omega)x^{m-1}
\end{align*}
mit
\begin{align*}
  r^{m+1} &= \omega r^m-\omega Ar^m+(1-\omega)r^{m-1}, \\
  p_{m+1} &= \omega(1-t)p_m(t)+(1-\omega)p_{m-1}(t).
\end{align*}
Daraus folgt

\begin{equation*}
  \begin{pmatrix}
    p_{m+1}(t) \\ p_m(t)
  \end{pmatrix}
  =
  \underbrace{
    \begin{pmatrix}
      \omega(1-t) & 1-\omega\\
      1 & 0
    \end{pmatrix}
  }_{B_\omega(t)}
  \begin{pmatrix}
    p_{m}(t) \\ p_{m-1}(t)
  \end{pmatrix} .
\end{equation*}
Wir wollen nun das \glqq{}optimale\grqq{} $\omega$ ermitteln,
d.h.\ wir suchen die L�sung  $\omega$ von
\begin{equation}
  \label{Richardsonmin_eq}
  \min_{\omega\in\rr\mathstrut}\,
  \max_{t\in[1-\gamma,1+\gamma]}
  \rho\bigl(B_\omega(t)\bigr).
\end{equation}

Zur besseren Darstellung machen wir den �bergang $t \to 1-t$, also
\begin{eqnarray*}
  1-t &\lr& t \in [-\gamma,\gamma],
  \\
  B_\omega(t)
  &=&
  \begin{pmatrix}
    \omega(1-t) & 1-\omega\\
    1 & 0
  \end{pmatrix}
  \lr
  \begin{pmatrix}
    \omega t & 1-\omega\\
    1 & 0
  \end{pmatrix}.
\end{eqnarray*}
Wir suchen damit die L�sung $\omega$ der Minimierungsaufgabe
\begin{equation*}
  \min_{\omega\in\rr\mathstrut}\,
  \max_{t\in[-\gamma,\gamma]}
    \rho\bigl(B_\omega(t)\bigr).
\end{equation*}

Nun bestimmen wir die Eigenwerte $\lambda$ von $B_\omega(t)$. Es gilt n�mlich
\begin{align*}
  (\lambda-\omega t)\lambda-(1-\omega) &= 0 \\
  \iff
  \lambda^2-(\omega t)\lambda-(1-\omega) &= 0,
\end{align*}
d.h.\ wir erhalten
\begin{align*}
  \lambda_{1,2}
  &=
  \frac{\omega t}{2}
  \pm
  \sqrt{
    \frac {\omega^2t^2} {4} + (1-\omega)
  }
  \,.
\end{align*}

Wir machen zun�chst eine
Fallunterscheidung zur Berechnung von $\rho\bigl(B_\omega(t)\bigr)$.

\textit{Fall I}: $\dfrac{\omega^2t^2}{4}+(1-\omega) < 0$.
Dann gilt
\begin{align*}
  \abs{\lambda_{1,2}}^2
  &=
  \frac{(\omega t)^2}{4}+\frac{(\omega t)^2}{4}+(1-\omega) \\
  &=
  \frac{(\omega t)^2}{2}+(1-\omega). \\
\end{align*}
Also gilt
\begin{equation}
  \label{Richardson:FallI_eq}
  \rho\bigl(B_\omega(t)\bigr)
  =
  \biggl(
  \frac {(\omega t)^2} {2} + (1-\omega)
  \biggr)^{\!\! 1/2}.
\end{equation}

\textit{Fall II}: $\dfrac{\omega^2t^2}{4}+(1-\omega)\geq0$.

(a) $\omega t\geq 0$: In diesem Fall haben wir
\begin{equation*}
  \rho\bigl(B_\omega(t)\bigr)
  =
  \frac{\omega t}{2}+\sqrt{\frac{\omega^2t^2}{4}+(1-\omega)}.
\end{equation*}

(b) $\omega t\leq 0$: Dann gilt
\begin{equation*}
  \rho\bigl(B_\omega(t)\bigr)
  =
  \abs{\frac{\omega t}{2}-\sqrt{\frac{\omega^2t^2}{4}+(1-\omega)}}.
\end{equation*}

In beiden F�llen (a) und (b) erhalten wir aber
\begin{equation}
  \label{Richardson:FallII_eq}
  \rho\bigl(B_\omega(t)\bigr)
  =
  \frac {\abs{\omega t}} {2} + \sqrt{\frac{\omega^2t^2}{4}+(1-\omega)}.
\end{equation}


Sei nun $\omega$ fest. Wir bestimmen
$\max\limits_{t\in[-\gamma,\gamma]}\rho\bigl(B_\omega(t)\bigr)=:\beta_\omega$.

\textit{1.~Fall}: $\omega \leq 1$. Es kommt nur Fall~II vor.  Aus
\eqref{Richardson:FallII_eq} erhalten wir, da $\rho\left(B_\omega(t)\right)$ eine
gerade Funktion ist und monoton wachsend f�r $t\in[0,\gamma]$
\begin{equation*}
  \beta_\omega
  =
  \frac {\abs{\omega}\gamma} {2}
  + \sqrt { \frac {\omega^2\gamma^2} {4} + (1-\omega) }.
\end{equation*}

\textit{2.~Fall}: $\omega>1$. Jetzt kommen Fall~I und Fall~II beide vor.
Es gilt
\begin{equation*}
  \frac {\omega^2t^2} {4} + (1-\omega)
  \geq 0
  \iff
  \abs{t} \geq \frac {2\sqrt{\omega-1}} {\omega}.
\end{equation*}
Dann ist
\begin{equation*}
  \rho\bigl(B_\omega(t)\bigr)
  =
  \frac {\omega \abs{t}} {2}
  + \sqrt { \frac {\omega^2t^2} {4} + (1-\omega) }.
\end{equation*}

Im Fall $\abs{t}\leq \dfrac{2\sqrt{\omega-1}}{\omega}$ ist
\begin{equation*}
  \rho\bigl(B_\omega(t)\bigr)
  =
  \left(
    \frac {(\omega t)^2} {2} + (1-\omega)
  \right)^{\!\!\! 1/2}\negthickspace,
\end{equation*}
also ist
\begin{multline*}
  \beta_\omega
  =
  \max
  \Biggl\{
    \max_{\abs{t}\leq\frac{2\sqrt{\omega-1}}{\omega}}
    \left(
      \frac {(\omega t)^2} {2} + (1-\omega)
    \right)^{\!\!\! 1/2}
    \negthickspace,
    \\
    \max_{\frac{2\sqrt{\omega-1}}{\omega}\leq\abs{t}\leq\gamma}
    \left(
      \frac {\omega\abs{t}} {2}
      + \sqrt { \frac {(\omega t)^2} {4} + (1-\omega) \! } \:
    \right)
  \Biggr\}.
\end{multline*}

Hierin kommt der erste Term nur vor, wenn $\gamma\geq
\frac{2\sqrt{\omega-1}}{\omega}$. Aus Monotoniegr�nden folgt
\begin{align*}
  \max_{\abs{t}\leq\frac{2\sqrt{\omega-1}}{\omega}}
  \left(\frac{(\omega t)^2}{2}+(1-\omega)\right)^{\!\!\! 1/2}
  &=
  \left(
    \frac {\bigl( \omega \frac {2\sqrt{\omega-1}} {\omega} \bigr)^2} {2}
    + (1-\omega)
  \right)^{\!\!\! 1/2}
  \\
  &=
  \sqrt{\omega-1}
  ,
  \\
  \max_{\frac{2\sqrt{\omega-1}}{\omega}\leq\abs{t}\leq\gamma}
  \left(
    \frac {\omega\abs{t}} {2}
    + \sqrt { \frac {(\omega t)^2} {4} + (1-\omega) \! } \:
  \right)
  &=
  \frac{\omega\gamma}{2}
  + \sqrt{ \frac {(\omega\gamma)^2} {4} + (1-\omega) }
  .
\end{align*}
F�r $\gamma\geq\frac{2\sqrt{\omega-1}}{\omega}$ gilt
\begin{align*}
  \frac{\omega\gamma}{2}
  + \sqrt{ \frac {(\omega\gamma)^2} {4} + (1-\omega) }
  &\geq
  \frac {\,\omega\frac{2\sqrt{\omega-1}}{\omega}\,} {2}
  +
  \sqrt {
    \frac {\bigl(\omega\frac{2\sqrt{\omega-1}}{\omega}\bigr)^2} {4}
    + (1-\omega)
  }
  \\
  &=
  \sqrt{\omega-1}
  .
\end{align*}
Damit gilt
\begin{equation*}
  \beta_\omega
  =
  \begin{cases}
    \sqrt{\omega-1}
    & \text{falls }
    \gamma<\frac{2\sqrt{\omega-1}}{\omega},
    \\
    \dfrac{\omega\gamma}{2}+\sqrt{\dfrac{(\omega\gamma)^2}4{}+(1-\omega)}
    & \text{falls }
    \gamma \geq \frac{2\sqrt{\omega-1}}{\omega}.
  \end{cases}
\end{equation*}
Unter Beachtung von
\begin{align*}
  &\gamma < \frac{2\sqrt{\omega-1}}{\omega} \\
  \iff&
  \omega^2\gamma^2 < 4(\omega-1) \\
  \iff&
  \frac{2-2\sqrt{1-\gamma^2}}{\gamma^2}
  < \omega <
  \frac{2+2\sqrt{1-\gamma^2}}{\gamma^2} \\
  \iff&
  \frac{2}{1+\sqrt{1-\gamma^2}}
  < \omega <
  \frac{2}{1-\sqrt{1-\gamma^2}}
\end{align*}
erhalten wir durch Zusammenfassen von "`1.~Fall"' und "`2.~Fall"'

\begin{equation*}
  \beta_\omega
  =
  \begin{cases}
    \sqrt{\omega-1}
    & \text{falls }
    \frac {2} {1+\sqrt{1-\gamma^2}} < \omega < \frac {2} {1-\sqrt{1-\gamma^2}}
    ,
    \\
    \frac{\omega\gamma}{2}+\sqrt{\frac{(\omega \gamma)^2}{4}+(1-\omega)}
    & \text{sonst.}
  \end{cases}
\end{equation*}

F�r $\omega\geq 2/\bigl(1-\sqrt{1-\gamma^2\!}\:\bigr)\;(\geq2)$
ist $\frac{\omega\gamma}{2}\geq \sqrt{\omega-1}\geq 1$, also
$\beta_\omega\geq 1$.
F�r $\omega < 2/\bigl(1+\sqrt{1-\gamma^2\!}\:\bigr)$ ist
\begin{equation*}
  \beta_\omega
  =
  \frac{\omega\gamma}{2}
  + \sqrt{\frac{(\omega\gamma)^2}{4}+(1-\omega)}
\end{equation*}
monoton fallend. Die Gr��e $\beta_\omega$ wird also minimal f�r
$\omega=\omega_0=\frac{2}{1+\sqrt{1-\gamma^2}}$ und hat dann den Wert
$\sqrt{\omega_0-1}$.

Zusammenfassend erhalten wir also:
\begin{sa}
  Der optimale Parameter $\omega$ f�r das Richardson-Verfahren
  2.~Ordnung, welcher die Minimierungsaufgabe \eqref{Richardsonmin_eq}
  l�st, ist
  \begin{equation*}
    \omega_0
    =
    \frac {2} {1+ \sqrt{1-\gamma^2}}
    \,.
  \end{equation*}
\end{sa}

Jetzt kann man f\"ur $ w = w_0 $ die Gr\"o\ss{}e

\[
\underset{t \in \left[ 1-\gamma , 1 + \gamma \right] }{\max} \left| p_m(t) \right|
\]

diskutieren unter Beachtung von $p_0 = 1$, $p_1(t) = 1-t$.

\begin{aufg} Bestimme f\"ur alle $t \in \left[ 1 - \gamma, 1 + \gamma \right] $ die L\"osung der
Differenzengleichung 2.~Ordnung
\[
p_{m+1}(t) = w_0 \cdot (1-t) p_m(t) + (1-w_0) p_{m-1}(t)
\]
unter den Anfangsbedingungen $ p_0 = 0$ und $ p_1(t) = 1-t $ und zeige damit
\begin{enumerate}
\item
$
\underset{t \in \left[ 1-\gamma , 1 + \gamma \right] }{\max} \left| p_m(t) \right| =
\left( \sqrt{w_0 - 1} \right)^m (1+m\sqrt{1-\gamma^2}) =: \rho_m,
$
(Dies ist sehr aufwendig!\footnote{Man liest also besser bei den Meistern nach:
G. Golub, R. Varga: Chebyshev semi-iterative methods, successive overrelaxation iterative
methods and second-order Richardson iterative methods, Part I and II, Numer. Math. {\bf 3},
 147-168 (1961).})
\item $ \rho_m $ f\"allt monoton in $m$,
\item $ \rho_m > c_m $,  $c_m = \frac{2c^{m}}{1+c^{2m}}$
aus dem Tschebyscheff Verfahren mit $c = \frac{\sqrt{\kappa} - 1}{\sqrt{\kappa}+1}$, $\kappa = \frac{1+\gamma}{1-\gamma} $,
\item es gilt $ \limn \frac{\rho_m}{c_m} = 1 $.
\end{enumerate}
\end{aufg}

\begin{bem}
Aus $ r^m = p_m(A) r^0$ folgt $e^m =  p_m(A) e^0$ mit $e^m = A^{-1}b - x^m$ und damit
\[
\|e^m\| \leq \| p_m(A) \| \|e^0\|.
\]
\end{bem}
Tschebyscheff- und Richardson-Verfahren machen eine Schranke
f"ur $\|p_m(A)\|$ klein. Dies wirkt sich demnach in den  Absch"atzungen
nicht nur f\"ur das Residuum, sondern auch
f"ur den {\em Fehler} aus.

%%% Local Variables:
%%% mode: latex
%%% TeX-master: "Iterationsverfahren03"
%%% End:
