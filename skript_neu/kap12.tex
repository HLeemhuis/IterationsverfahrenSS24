% Dieses Kapitel habe ich im Sommer 2007 neu geschrieben,
% weil mit dem Paper von Liessen, Faber und Tichy nun ein besserer
% Beweis fr den Faber-Manteuffel-Satz vorliegt
%
% Andreas Frommer, 22.6.2007


\newcommand{\kernel}{\mbox{Kern}}
\section{Minimales Residuum vs kurze Rekursion}

Das zentrale Resultat dieses Kapitels ist der Satz von Faber und Manteuffel,
Satz~\ref{faber:sa}. Zur Vorbereitung wiederholen wir ein paar 
Tatsachen "uber lineare Operatoren auf endlich-dimensionalen R"aumen, wie man sie aus
den Anf"angervorlesungen zur Linearen Algebra kennt. $A$ bezeichnet einen 
linearen Operator auf einem solchen Raum $V$ "uber $\MathC$, der mit einem Innenprodukt
$\langle \cdot, \cdot \rangle$ ausgestattet ist.

\begin{defn} 
 Die Adjungierte $A^*$ zu $A$ bez"uglich $\langle \cdot , \cdot \rangle$ ist die Matrix mit
\[
 \langle Ax , y  \rangle = \langle x, A^*y\rangle \quad \forall \; x,y \in V.
\]
\end{defn}

\begin{lem} \label{normal_eq:lem}
Die folgenden Bedingungen sind "aquivalent:
\begin{itemize}
 \item[(i)] $AA^* =  A^*A $,
 \item[(ii)] $V$ besitzt eine Orthonormalbasis aus Eigenvektoren $q_i$ von $A$ (zu Eigenwerten $\lambda_i$),
 \item[(iii)] $A^*=p(A), \quad p \in \Pi_{n-1}$, wobei $n = \dim(V)$.
\end{itemize}
\end{lem}
\begin{proof}
 (i) $\Leftrightarrow$ (ii) ist bekannt aus LA.

 (ii) $\Longrightarrow$ (iii): Aus $\langle q_j, A^*q_i \rangle = \langle Aq_j, q_i \rangle 
  = \lambda_j \delta_{ij}$ folgt $A^*q_i = \overline{\lambda}_i q_i$.
 Daraus folgt $A^* = p(A)$, wobei $p$ das Polynom aus
$\Pi_k$ ist mit $p(\lambda_i)=\overline{\lambda_i}, \quad i=1, \dots,k$. Dabei sind  $\lambda_1,\dots,\lambda_k$ die {\em verschiedenen}
Eigenwerte von $A$.

 (iii) $\Longrightarrow$ (i): trivial.
\end{proof}
%

\begin{defn} 
Ein linearer Operator, der eine der drei Bedingungen aus Lemma~\ref{normal_eq:lem}
erf"ullt, hei"st {\em normal}. $A$ hei"st {\em $s$-normal} ($s \in \mathbb{N}, s < \dim(V)$), falls 
$s$ der kleinste Grad von allen Polynomen ist, die 
Lemma~\ref{normal_eq:lem} (iii) erf"ullen.
\end{defn}
\medskip

Beispiel: $A$ selbstadjungiert $\Rightarrow$ $A$ ist 1-normal.
\medskip

\begin{aufg} Charakterisiere alle 1-normalen Operatoren.
\end{aufg}

\begin{defn}
$\mu_A$ und $d_A$ bezeichnen das Minimalpolynom von $A$ und dessen Grad. 
\\
F"ur $v \in V$ bezeichnen $\mu_v$ und $d_v$ das Minimalpolynom von $v$ bzgl.\ $A$
und dessen Grad. D.h., $\mu_v$ ist das Polynom kleinsten Grades mit 
$\mu_v(A)v = 0$, normiert auf H"ochstkoeffizient 1.
\end{defn}

\begin{lem} Gilt $p(A)v = 0$, so folgt $\mu_v \mid p$. Insbesondere
gilt $\mu_v \mid \mu_A$.
\end{lem}  
\begin{proof} Alle Polynome mit $p(A)v = 0$ bilden ein Ideal. Weil
der Polynomring ein Hauptidealring ist, liegen sie alle in dem 
von $\mu_v$ erzeugten Hauptideal.
\end{proof}

\begin{defn} Der von $v \in V$ erzeugte {\em zyklische Unterraum} ist
\[
   K(A,v) = K_{d_v}(A,v).
\]
Beachte: $d_v$ ist der Stufenindex, ab dem $K_m(A,v)$ stagniert.
\end{defn}

\begin{lem} \label{normal_tat1:lem}
Tatsachen "uber zyklische Unterr"aume:
\begin{itemize}
\item[(i)] $\hat{A}: K(A,v) \to K(A,v), \, w \to Aw$ ist linearer Operator auf $K(A,v)$.
\item[(ii)] F"ur $ w \in K(A,v)$ ist $K(A,w) \subseteq K(A,v)$. Gleichheit gilt genau dann,
          wenn $w = p(A)v$ mit $p$ ist teilerfremd zu $\mu_v$. 
\item[(iii)] Eigenr"aume in $K(A,v)$ haben immer Dimension 1.
\end{itemize}
\end{lem}
\begin{proof}
(i) ist klar; ebenso der erste Teil von (ii). F"ur $w = p(A)v$ gilt wegen $K(A,w) \subseteq 
K(A,v)$  einerseits $\mu_w \mid \mu_v$. Andererseits gilt
 $\mu_v \mid p \cdot \mu_w$. Dies beweist den Rest von (ii). Ein Eigenvektor $p(A)v$ zum Eigenwert $\lambda$
erf"ullt $(Ap(A)-\lambda p(A))v = 0$ mit $\deg p \leq d_v-1$. Also ist sogar $\mu_v = \alpha \cdot (t-\lambda) \cdot p(t)$,
was den Eigenvektor bis auf skalare Vielfache eindeutig festlegt.
\end{proof}

\begin{lem} \label{normal_tat2:lem}
Weitere Tatsachen:
\begin{itemize}
\item[(i)] Mit $\mu_A(t) = \prod_{i=1}^\ell (t-\lambda_i)^{e_i}$ gilt
\[
V = \oplus_{i=1}^\ell \kernel(A-\lambda_i I)^{e_i}
\]
\item[(ii)] Zu jedem $p \in \Pi$ mit $p \mid \mu_A$ existiert $u \in V$ mit $\mu_u = p$.
\end{itemize}
\end{lem}
\begin{proof} (i) ist bekannt. In (ii) nimmt man f"ur $p = \prod_{i=1}^\ell (t-\lambda_i)^{f_i}$ mit $f_i 
\leq e_i$ den Vektor $u = \sum_{f_i > 0} u_i$ mit $u_i \in \kernel(A-\lambda_i I)^{f_i} -
 \kernel(A-\lambda_i I)^{f_i-1}$.
\end{proof}
\medskip

Wir ben"otigen zwei technische Lemmas. 

\begin{lem} \label{FMproof1:lem}
$A,B: V \to V$ seien linear, $A$ invertierbar. Weiter sei $s+2 \leq d_A$ und 
f"ur alle $v \in V$ mit $d_v = d_A$  gelte
\[
Bv \in \spann(v,A,\ldots,A^s v).
\]
\begin{itemize}
\item[(i)] Dann gilt $AB = BA$.
\item[(ii)] Im Fall $B = A^*$ folgt: $A$ ist $t$-normal mit $t \leq s$.
\end{itemize} 
%Im Fall $B = A^*$ gilt sogar, dass dann $A$ $t$-normal ist
%mit $t \leq s$.
\end{lem}
\begin{proof}
(i): Sei $d_v = d_A$. F"ur $\gamma \not \in \spek(A)$ folgt mit $w_\gamma = (A-\gamma I)v$
aus Lemma~\ref{normal_tat1:lem}, dass $K(A,v) = K(A,w_\gamma)$ und $d_v = d_{w_\gamma}$; insbesondere gilt das f"ur $w_0 = Av$. Nach Voraussetzung existieren Polynome $p_\gamma,
q,r \in \Pi_s$ mit
\begin{equation} \label{BAs:eq}
   Bw_\gamma = p_\gamma(A)w, \; B(Av) = q(A)(Av), \; Bv = r(A)v,
\end{equation}
also
\[
Bw_\gamma = (p_\gamma(A)\cdot(A -\gamma I))v = B(Av)-\gamma Bv = (q(A)\cdot A - \gamma r(A))v. 
\]
F"ur $\gamma \not \in \spek(A)$ erf"ullt das Polynom $\phi_\gamma(t) = t\cdot(p_\gamma(t)-q(t))-\gamma(p_\gamma(t)-r(t)) \in \Pi_{s+1}$ 
also $\phi_\gamma(A)v = 0$, ist also Vielfaches von $\mu_v$. Wegen $ s+1 < d_v = d_A$
folgt $\phi_\gamma \equiv 0$. Hieraus folgt 
\[
   \gamma(q-r) = (t-\gamma)(p_\gamma - q),
\]
d.h.\ jedes $0 \neq \gamma \not \in \spek(A)$ ist Nullstelle von $q-r$,
also $q=r$ und damit nach \eqnref{BAs:eq} $BAv = q(A)Av = ABv = Ar(A)v$. 

Die Operatoren
$A$ und $B$  kommutieren also auf allen Vektoren $v$ mit $d_v = d_A$. F"ur einen gegebenen solchen Vektor
ist $\mbox{dist}(A^{d_v-1}v,\spann(v,\ldots,A^{d_v-2}v)) >0$; aus Stetigkeitsgr"unden 
gilt das dann auch noch, wenn man $v$ durch ein beliebiges $w$ aus einer gen"ugend kleinen 
Kugel um $v$ ersetzt. F"ur alle solche $w$ ist also $d_w = d_v$; insbesondere gibt
es eine Basis $b_1,\ldots,b_n$ von $V$ mit $d_{b_i} = d_v$. Gezeigt wurde bereits
$ABb_i = BAb_i$, also gilt $AB = BA$. \smallskip \\
(ii): Nach (i) ist $A$ normal, also $A^* = p(A)$ mit $\deg p \leq d_A-1$. F"ur ein $v$ mit 
$d_v = d_A$ gilt aber $A^*v = p(A)v \in \spann(v,Av,\ldots,A^sv)$. Weil $v,Av,\ldots,A^{d_A-1}v$
l.u.\ sind, folgt daraus $\deg p \leq s$.
\end{proof}
\medskip

Jetzt betrachten wir das Arnoldi-Verfahren \ref{Arnoldi-Verfahren} in
Abh"angigkeit vom Startvektor $v = h_{1,0}v_1 $. Zur Erinnerung: Man berechnet
eine ONB $\{v_1,v_2,\ldots\}$ durch
\begin{equation} \label{Arnoldi_rek.eq}
h_{i+1,i} v_{i+1} = Av_i - \sum_{j=1}^{i} h_{j,i}v_j.
\end{equation}
F"ur gegebenen Startvektor $u,v,w,x,\ldots$ bezeichnen dann $u_i,v_i,w_i,x_i, \ldots$
die entsprechenden Arnoldi-Vektoren. 

\begin{lem} \label{FMproof2:lem}
$A: V \to V$ sei linear und invertierbar. Es sei $1 < i  < n \leq d_A$.
F"ur alle $u \in V$ mit $d_u = n$ gelte $\langle u, Au_i \rangle =0$. Dann gilt
auch f"ur alle $v$ mit $i \leq d_v \leq n$ die Beziehung $\langle v, Av_i \rangle = 0$.
\end{lem}
\begin{proof}
Sei $v \in V$ mit $i \leq  d_v < n$. Sei $u \in V$ mit $d_u = n$ so, dass $\mu_v \mid \mu_u$. 
(Erg"anze $\mu_v$ um $d_u-d_v$ Linearfaktoren aus Eigenwerten und verwende Lemma \ref{normal_tat2:lem}(ii).)  Setze
\[
x_\gamma = v - \gamma u.
\]
Klar: $\mu_{x_\gamma} \mid \mu_u$. Angenommen, $\mu_{x_\gamma}$ ist echter Teiler
von $\mu_u$. Dann gilt
\begin{equation} \label{mugamma:eq}
  \mu_{x_\gamma}(A)v = \gamma \underbrace{\mu_{x_\gamma}(A)u}_{\neq 0}.
\end{equation}
Es gibt nur endlich viele Teiler von $\mu_u$. F"ur jeden solchen Teiler $\mu_{x_\gamma}$
wird \eqnref{mugamma:eq} f"ur nur ein $\gamma$ erf"ullt. F"ur alle bis auf endlich viele
$\gamma$ ist also $\mu_{x_\gamma} = \mu_u$. Sei jetzt $x = x_\gamma$ f"ur ein $\gamma$,
das keine Ausnahme ist. Dann ist im Arnoldi-Verfahren $x_i = p(A)x$ mit $p \in \Pi_{i-1}$
und es gilt f"ur $j=0,\ldots,i-2$
\[
  0 = \langle A^j x, x_i \rangle = \langle A^j x, p(A)x \rangle
\]
und, nach Voraussetzung,
\[
\langle x , Ax_i \rangle  = \langle x, Ap(A)x \rangle = 0.
\]
Dies sind $i$ Gleichungen eines homogenen LGS f"ur die $i$ Koeffizienten 
von $p \neq 0$, d.h. die Determinante
\[
\delta(x) = 
\left| 
\begin{array}{cccc}
   \langle x, x \rangle & \langle x, Ax \rangle & \ldots & \langle x, A^{i-1} x \rangle \\
  \langle Ax, x \rangle & \langle Ax, Ax \rangle & \ldots & \langle Ax, A^{i-1} x \rangle \\
    \vdots              &    \vdots              &    \ldots & \vdots \\
 \langle A^{i-2}x, x \rangle & \langle A^{i-2}x, Ax \rangle & \ldots & \langle A^{i-2}x, A^{i-1} x \rangle \\
 \langle x, Ax \rangle & \langle x, A^2x \rangle & \ldots & \langle x, A^{i} x \rangle 
\end{array}
              \right|
\]
verschwindet f"ur $x = x_\gamma$ f"ur alle bis auf endlich viele und damit f"ur alle
$\gamma$. Also existiert auch f"ur $\gamma = 0$, also $x=v$ eine nicht-triviale L"osung
des zugeh"origen LGS in Gestalt eines Polynoms $0 \neq q \in \Pi_{i-1}$. Die ersten
$i-1$ Gleichungen besagen, dass $0 \neq w = q(A)v$ orthogonal zu
$\spann(v,Av,\ldots,A^{i-2} v)$,
also Vielfaches von $v_{i}$ ist. Die letzte Gleichung besagt $\langle v, Aw \rangle = 0$,
also $\langle v, Av_i \rangle = 0$. 
\end{proof}


\begin{defn} $A$ {\em erlaubt eine $s$-stellige Rekursion}, wenn im
Arnoldi-Verfahren f"ur jeden Startvektor $v$ gilt $h_{j,i} = 0$ f"ur
$j < i-s$ und es einen Startvektor gibt, f"ur den f"ur wenigstens
ein $i$ die Zahl $h_{i-s,i} \neq 0$. 
\end{defn}

Insbesondere gilt also: $\langle v_j, Av_i\rangle = 0$ f\"ur $j=1,\ldots,i-s-1$.

\begin{sa}[Faber, Manteuffel 1982] 
\label{faber:sa}
$A$ erlaubt eine $s$-stellige Rekursion mit $s+2 \leq d_A$
 $\Longleftrightarrow$ $A$ ist $s$-normal. 
\end{sa}
\begin{proof}
"`$\Leftarrow$"': Es ist $h_{i,j} = \langle Av_i,v_j \rangle = \langle v_i, A^*v_j \rangle$
mit $A^* = p(A), p \in \Pi_s$. Also
\[
\langle v_i, \underbrace{p(A)v_j}_{\in K_{j+s}} \rangle = 0 \mbox{ f"ur } j+s \leq i-1.
\]
"`$\Rightarrow$"' (der harte Teil!): \\
{\em Teil I: Einschr"ankung auf $K(A,v)$ mit $d_v = s+2$.}\\
Nach Voraussetzung gilt
\begin{equation} \label{FMproof:eq}
\langle v, Av_{s+2} \rangle = 0 \enspace \mbox{f"ur alle } v \in V \mbox{ mit } d_v \geq s+3.
\end{equation}
Sei $u \in V$ mit $d_u = s+2$. Sei $\hat{A}: K(A,u) \to K(A,u), \, u \to Au$. Es
ist $d_u = d_{\hat{A}} = \dim K(A,u)$. Sei $y \in K(A,u)$ mit $d_y = d_u$
(bzgl.\ $A$ wie $\hat{A}$). Wegen \eqnref{FMproof:eq} und Lemma~\ref{FMproof2:lem}
gilt
\[
0 = \langle y, Ay_{s+2} \rangle =  \langle y, \hat{A}y_{s+2} \rangle = 
\langle \hat{A}^*y, y_{s+2} \rangle .
\]
Also gilt $\hat{A}^*y \in \spann(y,\hat{A}y,\ldots,\hat{A}^s y)$. Nach Lemma~\ref{FMproof1:lem}(ii)
folgt: $\hat{A}$ ist $t$-normal mit $t \leq s = \dim K(A,u) -1$. Insbesondere gilt mit paarweise
verschiedenen $\lambda_i$
\[
  K(\hat{A},y) = \kernel(\hat{A}-\lambda_1 I) \oplus_\perp \kernel(\hat{A}-\lambda_2 I) \oplus_\perp 
         \ldots \oplus_\perp \kernel(\hat{A}-\lambda_{s+2} I).
\]
(Alle Eigenr"aume sind 1-dimensional nach Lemma~\ref{normal_tat1:lem}(iii).).
\smallskip

{\em Teil II: "Ubertragung auf ganz $V$.} \\
Sei $v \in V$ mit $d_v = d_A$. Dann ist
\[
K(A,v) = \oplus_{i=1}^\ell K(A,z_i) \mbox{ mit } \mu_{z_i}(t) = (t-\lambda_i)^{c_i}.
\]
Wieder sind die $\lambda_i$ paarweise verschieden. OBdA seien sie nach absteigenden
$c_i$ nummeriert. Wegen $\dim(K(A,w)) = d_A > s+2$ existiert $m \leq \ell$ und $\tilde{c}_m
\leq c_m$ mit $c_1+\ldots + c_{m-1} + \tilde{c}_m = s+2$. Der Vektor $w = z_1 + \ldots +z_{m-1}
+ z_{m-1} + \tilde{z}_m$ mit $\tilde{z}_m \in \kernel(A-\lambda_m I)^{\tilde{c}_m} \setminus
\kernel(A-\lambda_m I)^{\tilde{c}_m-1} $ hat $d_w = s+2$. Nach Teil~I besitzt $A$ 
demnach $s+2$
verschiedene Eigenwerte auf $K(A,w)$, weshalb $c_1 = \ldots = \tilde{c}_m=1$ und damit
sogar $m=\ell$ und $c_1 = \ldots = c_\ell = 1$. Damit ist
\[
\mu_A(t) = \mu_w(t)  = \prod_{i=1}^\ell (t-\lambda_i)^1
\]
und deshalb (LA I)
\[
V = \oplus_{i=1}^\ell \kernel (A - \lambda_i I).
\]
Setze $V_i = \spann(z_i,\ldots,z_{s+i+1}) = K(A,\sum_{j=i}^{s+i+1} z_j)$
mit $z_j \in \kernel(A-\lambda_j I)$. Nach Teil I ist
\begin{equation} \label{orthog_faber_manteuffel:eq}
V_i = \kernel(A-\lambda_i I) \oplus_\perp \kernel(A-\lambda_{i+1} I) \oplus_\perp 
         \ldots \oplus_\perp \kernel(A-\lambda_{s+i+1} I),
\end{equation}
und es existiert ein Polynom $p_i \in \Pi_{s}$ mit $p_i(\lambda_j) = \overline{\lambda_j},
\, j=i,\ldots,s+i+1$. Die Polynome $p_i$ und $p_{i+1} \in \Pi_s$ stimmen in den $s+1$ Stellen $\lambda_{i+1}, \ldots, \lambda_{s+i+1}$ "uberein, sind also (alle!) identisch, $p_1=p_2
= \ldots = p_\ell = p$.
\\
Zusammengefasst: $A$ ist orthogonal diagonalisierbar (denn Eigenr\"aume zu $s+2$ beliebigen verschiedenen Eigenwerten sind orthogonal, weil in \eqref{orthog_faber_manteuffel:eq} die Reihenfolge der $\lambda_i$ beliebig sein darf)  und es existiert $p \in \Pi_s$ mit $p(\lambda) =
\overline{\lambda}$ f"ur alle $\lambda \in \spek(A)$. Also ist $A$ $s$-normal.
\end{proof}
\medskip

Frage: Haben wir auch gezeigt, dass $s$ nicht kleiner sein kann?
\medskip

Im Falle des Standard-Innenproduktes und der Standardbasis 
besitzt $A^*$ als Matrix die Darstellung
\[
A^* = \overline{A}^T.
\]

Wie ist dies bei einem beliebigen Innenprodukt (der Satz legt sich ja
nicht fest)? \\
Wechsel in der Notation: $\langle \; , \; \rangle_*$ ist jetzt ein beliebiges
Innenprodukt, $\langle \; , \; \rangle$ das "ubliche euklidische Innnenprodukt.

%
\begin{lem}
 Sei $\langle \; , \; \rangle_*$ ein Innenprodukt auf  $\cn$.
 Dann existiert ein $B \in \cnn$, $B$ hpd mit
\[\langle x , y \rangle_* = \langle x,B y \rangle \enspace (\mbox{$\langle \; , \; \rangle$ "ubliches Skalarprodukt).} \]
\end{lem}
\begin{proof}
 "Ubung.
\end{proof}
%
\begin{sa}
 Sei $\langle \; , \; \rangle_*$ gegeben und $B$ hpd mit $\langle x, y \rangle_*$ = $\langle x, By\rangle$. Dann gilt:  $A$ ist normal bzgl.
$\langle \; , \; \rangle_* \Longleftrightarrow B^{\frac{1}{2}}AB^{-\frac{1}{2}} $ ist normal bzgl. $\langle \; , \; \rangle$.
\end{sa}
\begin{proof}
 \begin{align*}
  && \langle Ax, y\rangle_*  & = \langle x, A^*y\rangle_* \\
  & \Longleftrightarrow &  \langle Ax, By\rangle & =  \langle x, BA^*y\rangle \\
  & \Longleftrightarrow &  \langle BAx, y\rangle & =  \langle x, BA^*y\rangle \quad \forall \; x,y \\
  & \Longleftrightarrow  & (BA^*)^H   & =  BA   \\
  & \Longleftrightarrow  & BA^* & =  A^HB  \\
  & \Longleftrightarrow & A^* & = B^{-1}A^HB.
 \end{align*}
Es ist also
 \begin{align*}
  && A^*A & =AA^* \\
  & \Longleftrightarrow & B^{-1}A^HBA  & = AB^{-1} A^HB  \\
  & \Longleftrightarrow  & B^{-\frac{1}{2}}A^H B^{\frac{1}{2}}B^{\frac{1}{2}}AB^{-\frac{1}{2}}  & = B^{\frac{1}{2}}AB^{-\frac{1}{2}}B^{-\frac{1}{2}}A^HB^{\frac{1}{2}}   \\
  & \Longleftrightarrow  & \tilde{A}^H \tilde{A}  & = \tilde{A} \tilde{A}^H
 \end{align*}
mit $\tilde{A}=B^{\frac{1}{2}}AB^{-\frac{1}{2}}$.
\end{proof}
%

\medskip

\textbf{Bezeichnung:} Statt "`$A$ normal bzgl. $\langle \; , \; \rangle_*$"' sagt man auch: "`$A$ ist $B$-normal"' ($B$ ist die zu
$\langle \; , \; \rangle_*$ geh"orige hpd-Matrix).
%
\begin{cor}
 $A$ ist $B$-normal $\Longleftrightarrow \exists \; p \in \Pi_n$ mit $A^*=p(A)$.
\end{cor}
\begin{proof}
 \begin{align*}
 A \text{ ist $B$-normal} & \Longrightarrow    & \tilde{A}    &= B^{\frac{1}{2}}AB^{-\frac{1}{2}} \text{ normal} \\
 & \Longrightarrow        & \tilde{A}^H     &= p(\tilde{A}) = B^{\frac{1}{2}}p(A)B^{-\frac{1}{2}} \\
 & \Longrightarrow        & B^{\frac{1}{2}}\tilde{A}^HB^{-\frac{1}{2}} &=p(A) \\
 & \Longrightarrow        & A^*        & = p(A)
 \end{align*}
\end{proof}
%

%% Vorlesung 18
%% Freitag, 27.6.2003
%% David Fritzsche

\textbf{Beachte:} Satz~\ref{faber:sa} setzt eine bestimmte Form f\"ur kurze
Rekursion voraus.

Alternative: z.B.~gekoppelte kurze Rekursion. Dann k\"onnten doch weitere
Verfahren mit minimalem Residuum und kurzer Rekursion existieren.

\begin{bsp}
  Sei $A\in\cnn$ unit\"ar, also $AA^H=I$. Es exisitiert eine kurze Rekursion
  zur Berechnung der Arnoldi-Vektoren
  \begin{equation*}
    h_{k,k+1}v^{k+1} = Av^k-\sum_{j=1}^{k}h_{k,j}v^j
    \qquad
    \text{mit }
    \norm{v^k}_2=1
    \text{ und }
    v^0=\frac{1}{\norm{r^0}}r^0.
  \end{equation*}
Die Vektoren
  $v^0,\dots,v^k$ sind eine Orthonormalbasis von $K_{k+1}(A,r^0)$ und
  {$Av^0,\dots,Av^k$} ist Orthonormalbasis von $AK_{k+1}(A,r^0)$.

  Damit ist $w^k,Av^0,\dots,Av^{k-1}$ bei geeigneter Wahl von $w^k$ eine
  Orthonormalbasis von $K_{k+1}(A,r^0)$.
  Eine geeignete Wahl ergibt sich induktiv "uber
  \begin{align*}
    \tilde w^k
    &= w^{k-1} - \sum_{j=0}^{k-1}
    \frac {\langle w^{k-1},Av^j\rangle} {\langle Av^j,Av^j\rangle} Av^j
    && (\langle Av^j,Av^j\rangle = 1)
    \\
    &= w^{k-1} - \sum_{j=0}^{k-1} \langle w^{k-1},Av^j\rangle Av^j
    && (w^{k-1}\perp Av^j \text{ f\"ur } j=0,\dots,k-2)
    \\
    &= w^{k-1} - \langle w^{k-1},Av^{k-1}\rangle Av^{k-1}
    &&
    \\
    w^k
    &= \frac{1}{\norm{\tilde w^k}}\tilde w^k
    .
    &&
  \end{align*}

  Die Berechnung von $v^{k+1}$ geht jetzt so:
  \begin{equation*}
    \tilde v^{k+1} = Av^k -\Bigl(\sum_{j=1}^{k}\beta_{k,j}Av^{j-1}\Bigr)
    - \beta_{k,0}w^k
  \end{equation*}
  \begin{align*}
    \text{mit}\qquad
    \beta_{k,j}
    &= \langle Av^k, Av^{j-1}\rangle = 0
    &&\text{da } v^k\perp v^{j-1}
    \\
    \beta_{k,0}
    &=
    \langle Av^k, w^k\rangle
    &&
  \end{align*}
  Wir erhalten also folgenden Algorithmus~\ref{alg:unitaeres-arnoldi-verf}
  zur Berechnung der Arnoldi-Vektoren.

  \begin{alg}[Unit\"ares Arnoldi-Verfahren]\label{alg:unitaeres-arnoldi-verf}
    \begin{algorithm}
      \begin{algorithmic}
        \STATE $v^0 := \bigl(1/\norm[normal]{r^0}\bigr)r^0$
        \STATE $w^0 := v^0$
        \FOR{$k=0,1,2,\dots$}
          \STATE $\beta_k = \langle Av^k,w^k\rangle$
          \STATE $\tilde v^{k+1} := Av^k-\beta_k w^k$
          \STATE $v^{k+1}
            :=\bigl(1/\norm[normal]{\tilde v^{k+1}}\bigr)\tilde v^{k+1}$
          \STATE $\tilde w^{k+1} := w^k-\overline{\beta}_k Av^k$
          \STATE $w^{k+1}
            :=\bigl(1/\norm[normal]{\tilde w^{k+1}}\bigr)\tilde w^{k+1}$
        \ENDFOR
      \end{algorithmic}
    \end{algorithm}
  \end{alg}

  Jetzt ist $H_{k,k}$ und $H_{k+1,k}$ aus dem Arnoldi-Prozess nicht
  mehr explizit gegeben. Man kann aber zeigen (Jagels und Reichel, 1994)
  \begin{enumerate}
  \item $H_{k,k}$ wird durch Algorithmus~\ref{alg:unitaeres-arnoldi-verf}
    in faktorisierter Form (als Produkt von Rotationen) implizit gegeben.
  \item F\"ur Systeme der Form
    \begin{equation*}
      (\zeta I+\rho A)x=b
    \end{equation*}
    mit $A^HA=I$ und $\zeta,\rho\in\co$ ist
    $K_m(\zeta I+\rho A,r^0)=K_m(A,r^0)$ und es existiert auch eine
    kurze Rekursion zur Bestimmung der Iterierten $x^k$ mit
    \begin{equation*}
      \norm[big]{b-(\zeta I+\rho A)x^k}_2
      =
      \min_{p_k\in\bar \Pi_k}
      \norm[big]{p_k(\zeta I+\rho A)b}_2
      .
    \end{equation*}
    Die Herleitung erfolgt wie bei MINRES unter Ausnutzung der faktorisierten
    Form von $H_{k+1,k}$
  \end{enumerate}

\end{bsp}




%%% Local Variables:
%%% mode: latex
%%% TeX-master: "Iterationsverfahren03"
%%% End:

