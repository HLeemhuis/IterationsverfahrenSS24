\section{L�sungs-Versuche}

\begin{aufg}
Finde ausgehend von Satz \nref{Minmaxkreis_sa} die L"osung von
\[
\underset{p_m\in \overline{\Pi}_m,}{\min}\ \underset{\lambda\in D(\gamma,\rho)}{\max}|p_m(\lambda)|.
\]
\end{aufg}
\textbf{Antwort:}
$p_m(z)=\frac{(z+\gamma)^m}{\gamma^m}$, wie man leicht mittels linearer Transformation $z\mapsto z-\gamma$
aus Satz \ref{Minmaxkreis_sa} einsieht.


\begin{aufg}
Bestimme jeweils die (im Sinne von Satz \nref{cmschranke_sa}) asymptotisch optimale L�sung der MinMax-Aufgabe
\[
\underset{p_m \in \overline{\Pi}_m }{\min} \quad \underset{\lambda \in E}{\max} |p(\lambda)|
\]
und den zugeh�rigen Wert des Minimums f�r die F�lle:
\begin{enumerate}
\item $E = a + bE_{\rho}$, wobei $a,b > 0$ und $ a - b\cdot \frac{1}{2}(\rho+1/\rho)  > 0 $
\begin{center}
\begin{picture}(200,100)
\put(150,50){\ellipse{80}{40}}
\put(150,30){\line(0,1){40}}
\put(0,50){\vector(1,0){200}}
\put(100,0){\vector(0,1){100}}
\end{picture}
\end{center}

\item $E = a+be^{i\frac{\pi}{2}}E_{\rho}$, wobei $a,b > 0$ und $ a - b \cdot \frac{1}{2}(\rho-1/\rho) > 0 $
\begin{center}
\begin{picture}(200,100)
\put(150,50){\ellipse{40}{80}}
\put(150,10){\line(0,1){80}}
\put(0,50){\vector(1,0){200}}
\put(100,0){\vector(0,1){100}}
\end{picture}
\end{center}

\item $E_\rho$ ist Geradenst�ck parallel zur reellen Achse von $-a$ bis $+a$ mit Achsenabschnitt b auf der imagin�ren Achse
\begin{center}
\begin{picture}(200,100)
\put(70,70){\line(1,0){60}}
\put(61,58){$-a$}
\put(121,58){$+a$}

\put(0,50){\vector(1,0){200}}
\put(100,0){\vector(0,1){100}}
\end{picture}
\end{center}
Sind in diesem Fall die Polynom auch optimal f"ur endliches $m$? 
\end{enumerate}
\end{aufg}
\textbf{Antwort:}
\begin{Blist}{1. und 2.}
\item[1. und 2.] wurden mit den Tschebyscheff-Polynomen behandelt.
\item[3.]
\end{Blist}

\begin{aufg}
Sei $A \neq A^H$, aber mit einer der folgenden speziellen Eigenschaften:
\begin{enumerate}
\item $A^H = \overline{A} \in \cnn$ (z.B. bei der Diskretisierung der Maxwell-Gleichungen in der Elektrodynamik)
\item $A^HJ = J^HA,\ J \in \cnn,\ J=J^H$ (man nennt $A$ dann auch {\em 
$J$-hermitesch})
\item $A^TJ = J^TA,\ J \in \cnn$ (man nennt $A$ dann auch {\em $J$-symmetrisch})
\end{enumerate}
Zeige jeweils: F�r geeignete Wahl von $\tilde{r}^0$ (bzw. $w^0$) kann man den unsymmetrischen Lanczos-Prozess mit nur einer MVM mit $A$
(und eventuell einer zus�tzlichen mit $J$) durchf�hren 
(dies �bertr�gt sich dann auch auf BiCG).
\end{aufg}
\begin{proof}
\begin{enumerate}
\item 
\[
AV^m = V^{m+1} T_{m+1,m} \quad \text{und} \quad A^H W^m = W^{m+1} \overline{T}_{m+1,m}
\]
$\Longrightarrow A \overline{W}^m = \overline{W}^{m+1} T_{m+1,m}$. W�hle also $w^0 = \overline{v}^0$, dann gilt: $w^m = \overline{v}^m \forall m$
\item
\[
AV^m = V^{m+1} T_{m+1,m} \quad \text{und} \quad A^H W^m = W^{m+1} \overline{T}_{m+1,m}
\]
Ansatz: $w^i = Jv^i$, dann gilt: \[A^H W_m = A^H J V_m = J A V_m = \underbrace{J V_{m+1}}_{=W_{m+1}}T_{m+1,m}.\] 
Es mu� also gelten das $T_{m+1,m} \in \mathbb{R}^{(m+1) \times m }$
\[
\alpha_m = \langle A v^m, w^m \rangle = \langle A v^m, J v^m \rangle = \langle J^H A v^m, v^m \rangle \in \mathbb{R}
\]
denn: $(J^H A)^H = A^H J=JA=J^HA$
\[
\delta_m = \langle \overline{v}^m, \overline{w}^m \rangle = \langle \overline{v}^m, J \overline{v}^m \rangle \in \mathbb{R}
\]
da $J=J^H$
\item analog zu 2 mit dem Ansatz: $w^i = \overline{J v^i}$
\end{enumerate}
\end{proof}
