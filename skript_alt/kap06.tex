\section{Basen f\"ur Krylov-Unterr\"aume}

Gegeben: $A \in \cnn$ nicht notwendigerweise hermitesch, $A$ regul�r,
\[
K_m = \langle r, Ar,...,A^{m-1}r \rangle.
\]

Gesucht: $v^1,...,v^m$ mit "`sch\"onen Eigenschaften"' und
\[
K_m = \langle v^1,...,v^m \rangle.
\]

{\bf Erinnerung:} Gram-Schmidt-Orthogonalisierung

Gegeben: $y_1,...,y_m \in \cnn$.

\begin{alg}[Gram-Schmidt-Orthogonalisierung, klassisch] \label{Gram-Schmidt-Verfahren instabil}
~\vspace*{-2\baselineskip}
\begin{algorithm}
  \begin{algorithmic}
    \STATE $v^i = y^i, \; i=1,\ldots,m$
    \FOR{$i = 1, \dots,m$}
      \FOR{$j = 1, \dots ,i-1$}
        \STATE $v^i = v^i - \langle y^i,v^j\rangle v^j$
      \ENDFOR
      \STATE $v^i = \frac{1}{\|v^i\|} \cdot v^i$
    \ENDFOR
  \end{algorithmic}
\end{algorithm}
\end{alg}

Es ist bekannt, dass diese Variante aber numerisch nicht stabil ist.
G"unstiger ist es, wenn man bei der Berechnung der Innenprodukte statt $y_i$
den aktuellen Vektor $v^i$ verwendet. Dies ist m"oglich, da zu jedem
Zeitpunkt der $j$-Schleife der Vektor $y^i-v^i \in \langle v^1,
\ldots,v^{j-1} \rangle$  senkrecht auf $v^j$ steht.

\begin{alg}[modifizierte Gram-Schmidt-Orthogonalisierung] \label{modifizieres Gram-Schmidt-Verfahren}
~                               % um "x.x Algorithmus" aus dem Kasten rauszubekommen
\vspace*{-2\baselineskip}       % um den Leeraum zu entfernen
\begin{algorithm}
  %\caption{modifiziertes Gram-Schmidt-Verfahren. }
  \begin{algorithmic}
    \STATE $v^i = y^i\; i=1,\ldots,m$
    \FOR{$i = 1, \dots,m$}
      \FOR{$j = 1, \dots ,i-1$}
        \STATE $v^i = v^i - \langle v^i,v^j\rangle v^j$
      \ENDFOR
      \STATE $v^i = \frac{1}{\|v^i\|} \cdot v^i$
    \ENDFOR
  \end{algorithmic}
\end{algorithm}
\end{alg}

Unser Ziel ist es, sukzessive eine ONB von $K_m(A,r)$ aufzubauen.

\medskip

{\bf Idee:}

$\langle v^1,...,v^m\rangle$ sei eine ONB in $K_m(A,r)$ mit $v^m \notin K_{m-1}(A,r)$. Dann
gilt insbesondere $A v^m \in K_{m+1}(A,r)\setminus K_m(A,r)$. Orthogonalisiere also $Av^m$ gegen $v^1,...,v^m$.

Durch Anwendung des modifizierten Gram-Schmidt-Verfahrens resultiert
das {\em Arnoldi-Verfahren}.

\begin{alg}[Arnoldi-Verfahren] \label{Arnoldi-Verfahren}
~  				% um "3.3 Algorithmus" aus dem Kasten rauszubekommen
\vspace*{-2\baselineskip}	% um den Leeraum zu entfernen
\begin{algorithm}
  %\caption{Arnoldi-Verfahren}
  \begin{algorithmic}
    \STATE $v^1 = r$, $h_{1,0} = \|r\|$, $v^1 = \frac{1}{h_{1,0}}r$
    \FOR{$i = 2, 3, \dots$}
      \STATE $v^i = Av^{i-1}$
      \FOR{$j = 1, \dots ,i-1$}
        \STATE $h_{j,i-1} = \langle v^i,v^j\rangle$
        \STATE $v^i = v^i - h_{j,i-1} v^j$
      \ENDFOR
      \STATE $h_{i,i-1} = \| v^i \|$
      \STATE $v^i = \frac{1}{h_{i,i-1}} \cdot v^i$
    \ENDFOR
  \end{algorithmic}
\end{algorithm}
\end{alg}

Wir f\"uhren folgende Kurznotation ein

\begin{equation}
\label{obereHessenbergmatrix} A V_m = V_{m+1} H_{m+1,m}
\end{equation}

mit

\[
V_m = \left[ v^1,...,v^m \right] \in \co^{n \times m}
\]

\[
H_{m+1,m}=\left(\begin{array}{cccc}
h_{1,1} & h_{1,2}& \cdots & h_{1,m}\\
h_{2,1} & h_{2,2} &\ddots & h_{2,m}\\
&\ddots&\ddots&\vdots \\
&&\ddots& h_{m,m}\\
&&& h_{m+1,m}
\end{array}\right) \in \co^{(m+1)\times m}.
\]

Im Falle $ A = A^H $ vereinfacht sich der Arnoldi-Prozess, da
die meisten Eintr"age in $H_{m+1,m}$ verschwinden. Es bezeichne
dazu $H_m$ die Matrix
\[
H_m = \left(\begin{array}{cccc}
h_{1,1} & h_{1,2}& \cdots & h_{1,m}\\
h_{2,1} & h_{2,2} &\ddots & h_{2,m}\\
&\ddots&\ddots&\vdots \\
&& h_{m-1,m} & h_{m,m}
\end{array}\right) \in \co^{m\times m}.
\]

\begin{lem}
Falls $A$ hermitesch ist, ist  $H_m$ eine reelle, symmetrische Tridiagonalmatrix.
Es gilt also $h_{i+1,i} = h_{i,i+1}$ f"ur alle $i$ und $h_{i,j} = 0 $ f"ur $i > j+1$.
\end{lem}

\begin{proof}
Aus \eqnref{obereHessenbergmatrix} folgt

\begin{align*}
V_m^H A V_m &= V_m^H V_{m+1} H_{m+1,m} \\
            &= \left(\begin{array}{cc}
                        & 0\\
                        & \vdots\\
\quad                        \raisebox{12pt}[-12pt]{\Large $I_m$} \quad& 0 \\
               \end{array}\right) H_{m+1,m} \\
            &= H_m.
\end{align*}
Mit $V_m^HAV_m$ ist auch $H_m$ hermitesch, besitzt also Tridiagonalgestalt.
Au"serdem sind dann die Diagonalelemente reell (und positiv). Die Elemente
$h_{i,i-1}$ der ersten Nebendiagonalen ergeben sich nach
Algorithmus~\ref{Arnoldi-Verfahren}  als $\|v^i\|$, sind also auch reell.
\end{proof}

Damit erhalten wir aus dem Arnoldi-Verfahren das Lanczos-Verfahren zur Bestimmung einer ONB von
$K_m(A,r)$ im Falle $A = A^H$.

\begin{alg}[Lanczos-Verfahren] \label{Lanczos-Verfahren0}
~  				% um "3.3 Algorithmus" aus dem Kasten rauszubekommen
\vspace*{-2\baselineskip}	% um den Leeraum zu entfernen
\begin{algorithm}
  %\caption{Lanczos-Verfahren}
  \begin{algorithmic}
    \STATE $v^1 = \frac{r}{\|r\|}, \; v^0 = 0,\; h_{1,0} = 0$
    \FOR{$i = 2,3, \dots$}
      \STATE $v^i = Av^{i-1}$
      \STATE $v^i = v^i - h_{i-1,i-2}v^{i-2}$
      \STATE $h_{i-1,i-1} = \langle v^i,v^{i-1}\rangle$
      \STATE $v^i = v^i - h_{i-1,i-1}v^{i-1}$
      \STATE $h_{i,i-1} = \|v_i\|$
      \STATE $v^i = \frac{1}{h_{i,i-1}}\cdot v^i$
    \ENDFOR
  \end{algorithmic}
\end{algorithm}
\end{alg}

Eine alternative Bezeichnung f"ur die Koeffizienten im Lanczos-Verfahren ist

\[
h_{i-1,i-2} = \beta_{i-1}, \enspace
h_{i-1,i-1} = \alpha_{i-1}.
\]

Damit l�sst sich der Algorithmus wie folgt formulieren:

\begin{alg}[Lanczos-Verfahren] \label{Lanczos-Verfahren}
~               % um "3.3 Algorithmus" aus dem Kasten rauszubekommen
\vspace*{-2\baselineskip}   % um den Leeraum zu entfernen
\begin{algorithm}
  %\caption{Lanczos-Verfahren}
  \begin{algorithmic}
    \STATE $v^1 = \frac{r}{\|r\|}, \; v^0 = 0,\; \beta_1 = 0$
    \FOR{$i = 2,3, \dots$}
      \STATE $q^i = Av^{i-1} - \beta_{i-1} v^{i-2}$
      \STATE $\alpha_{i-1} = \langle q^i,v^{i-1} \rangle$
      \STATE $q^i = q^i - \alpha_{i-1} v^{i-1}$
      \STATE $\beta_{i} = \|q^i\|$
      \STATE $v^i = \frac{q^i}{\beta_{i}}$
    \ENDFOR
  \end{algorithmic}
\end{algorithm}
\end{alg}

Hierin gilt:
\begin{equation}
\label{betamalalpha} \beta_{i} v^i = A v^{i-1} - \alpha_{i-1} v^{i-1} - \beta_{i-1} v^{i-2}.
\end{equation}

\begin{defn}
Zwei Folgen $v^1,\cdots ,v^m$ und $w^1,\cdots ,w^m \in \cn $ hei�en {\em bi-orthogonal},
falls gilt
\[
\langle v^i,w^j \rangle  = \delta_{i,j}.
\]
\end{defn}

\begin{bem}
$v^i,w^j$ sind nur bis auf reziproke Vielfache eindeutig bestimmt.
\end{bem}

Im Falle $A \neq A^H$ konstruieren wir nun eine bi-orthogonale Basis $v^1,\cdots ,v^m$ von $K_m(A,r)$ und
$w^1,\cdots ,w^m$ von $K_m(A^H,\tilde{r})$.

\bigskip

Man berechnet $v^1,w^1$ mit dem {\em unsymmetrischen} Lanczos-Verfahren.

\bigskip

{\bf Situation:}
\begin{quote}
$v^1,\ldots, v^m$ sei eine Basis von $K_m(A,r)$

$w^1,\ldots, w^m$ sei eine Basis von $K_m(A^H,\tilde{r})$

die beiden Basen sind bi-orthogonal
\end{quote}
und
\begin{quote}
$v^m \notin K_{m-1}(A,r)$

$w^m \notin K_{m-1}(A^H,\tilde{r})$.
\end{quote}

{\bf Ansatz:}
\begin{align*}
q^{m+1} &= Av^m,\\
p^{m+1} &= A^Hw^m.
\end{align*}
Wir setzen an:
\begin{align}
\tilde{v}^{m+1} &= q^{m+1} - \sum_{i=1}^m \alpha_i v^i,\\
\tilde{w}^{m+1} &= p^{m+1} - \sum_{i=1}^m \beta_i w^i.
\end{align}

{\bf Bedingung f"ur Bi-Orthogonalit"at:}
\begin{align*}
\langle \tilde v^{m+1},w^j \rangle =& \langle q^{m+1},w^j \rangle -\alpha_j \overset{!}{=} 0,\\
\langle v^{j}, \tilde w^{m+1} \rangle =& \langle v^j,p^{m+1} \rangle -\overline{\beta}_j \overset{!}{=} 0.
\end{align*}

Damit erhalten wir
\begin{align*}
\alpha_j =&\langle q^{m+1},w^j \rangle,\\
\beta_j =&\langle p^{m+1},v^j \rangle.
\end{align*}

F�r $ j \leq m-2$ gilt allerdings weiter
\begin{align*}
\alpha_j =\langle q^{m+1},w^j \rangle
                       &=\langle Av^m,w^j \rangle \\
                       &=\langle v^m,A^Hw^j \rangle \\
                       &=0,
\end{align*}
da $A^Hw^j \in K_{m-1}(A^H,\tilde{r})$. Analoges gilt f"ur die $\beta_j$, d.h.\ es
 gilt
\[
\alpha_j = \beta_j = 0 \mbox{ f"ur } j=1,\ldots,m-2.
\]

Zur Normierung sei
\begin{align*}
\delta_{m+1} &= \left( \langle \tilde{v}^{m+1},\tilde{w}^{m+1} \rangle \right) ^{\frac 12}\\
v^{m+1} &= \frac{1}{\delta_{m+1}}\tilde{v}^{m+1} \\
w^{m+1} &= \frac{1}{\overline{\delta}_{m+1}}\tilde{w}^{m+1}
\end{align*}
Schlie"slich weisen wir noch nach, dass die Koeffizienten $\alpha_{m-1}$ und
$\beta_{m-1}$ vom vorherigen Schritt schon bekannt sind (daf"ur verwenden wir
kurzfristig einen zus"atzlichen oberen Index):
\begin{align*}
\alpha_{m-1} &= \langle Av^m,w^{m-1}\rangle\\
 &= \langle v^m,A^Hw^{m-1} \rangle\\
 &= \langle v^m,\tilde{w}^m + \beta_{m-1}^{(m-1)} w^{m-1} + \beta_{m-2}^{(m-1)} w^{m-2} \rangle\\
 &= \langle v^m,\tilde{w}^m \rangle = \delta_{m}
\intertext{ }
\beta_{m-1} &= \langle A^Hw^m,v^{m-1} \rangle\\
 &= \langle w^m,Av^{m-1} \rangle \\
 &= ...\\
 &= \langle w^m,\tilde{v}^m\rangle = \overline{\delta}_{m}
\end{align*}

Insgesamt erhalten wir damit folgendes Verfahren:

\begin{alg}[unsymmetrisches Lanczos-Verfahren] \label{unsymmetrisches Lanczos-Verfahren}
~               % um "3.3 Algorithmus" aus dem Kasten rauszubekommen
\vspace*{-2\baselineskip}   % um den Leeraum zu entfernen
\begin{algorithm}
  %\caption{Lanczos-Verfahren}
  \begin{algorithmic}
    \STATE $v^0 = w^0 = 0$\\
    \STATE $\delta_1 = \left( \langle r, \tilde{r} \rangle \right)^{\frac 12} $
    \STATE $v^1 = \frac{1}{\delta_1} \cdot r$
    \STATE $w^1 = \frac{1}{\delta_1} \cdot \tilde{r}$
    \FOR{$m = 1,2, \dots$}
      \STATE $\alpha_m = \langle Av^m,w^m\rangle $
      \STATE $\tilde{v}^{m+1} = Av^m - \alpha_m v^m - {\delta}_m v^{m-1}$
      \STATE $\tilde{w}^{m+1} = A^Hw^m - \overline{\alpha}_m w^m - \overline{\delta_m} w^{m-1}$
      \STATE $\delta_{m+1} = \left( \langle \tilde{v}^{m+1},\tilde{w}^{m+1} \rangle \right )^{\frac 12} $
      \STATE $v^{m+1} = \frac{1}{\delta_{m+1}} \cdot \tilde{v}^{m+1}$
      \STATE $w^{m+1} = \frac{1}{\overline{\delta}_{m+1}} \cdot \tilde{w}^{m+1}$
    \ENDFOR
  \end{algorithmic}
\end{algorithm}
\end{alg}

\begin{bem}
Im unsymmetrischen Lanczos kann Folgendes passieren
\[
\delta_m = 0,
\]
obwohl $\tilde{v}^m \neq 0$ und $\tilde{w}^m \neq 0$.

\medskip

Das nennt man {\em serious breakdown}, da das Verfahren abbricht, ohne dass eine Basis f�r
den maximalen Krylov-Unterraum $K_{m^*}(A,r)$ oder $K_{m^{**}}(A^H,\tilde{r})$ berechnet
wurde.
\end{bem}
